\documentclass[11pt]{article}

    \usepackage[breakable]{tcolorbox}
    \usepackage{parskip} % Stop auto-indenting (to mimic markdown behaviour)
    
    \usepackage{ctex}
    % Basic figure setup, for now with no caption control since it's done
    % automatically by Pandoc (which extracts ![](path) syntax from Markdown).
    \usepackage{graphicx}
    % Keep aspect ratio if custom image width or height is specified
    \setkeys{Gin}{keepaspectratio}
    % Maintain compatibility with old templates. Remove in nbconvert 6.0
    \let\Oldincludegraphics\includegraphics
    % Ensure that by default, figures have no caption (until we provide a
    % proper Figure object with a Caption API and a way to capture that
    % in the conversion process - todo).
    \usepackage{caption}
    \DeclareCaptionFormat{nocaption}{}
    \captionsetup{format=nocaption,aboveskip=0pt,belowskip=0pt}

    \usepackage{float}
    \floatplacement{figure}{H} % forces figures to be placed at the correct location
    \usepackage{xcolor} % Allow colors to be defined
    \usepackage{enumerate} % Needed for markdown enumerations to work
    \usepackage{geometry} % Used to adjust the document margins
    \usepackage{amsmath} % Equations
    \usepackage{amssymb} % Equations
    \usepackage{textcomp} % defines textquotesingle
    % Hack from http://tex.stackexchange.com/a/47451/13684:
    \AtBeginDocument{%
        \def\PYZsq{\textquotesingle}% Upright quotes in Pygmentized code
    }
    \usepackage{upquote} % Upright quotes for verbatim code
    \usepackage{eurosym} % defines \euro

    \usepackage{iftex}
    \ifPDFTeX
        \usepackage[T1]{fontenc}
        \IfFileExists{alphabeta.sty}{
              \usepackage{alphabeta}
          }{
              \usepackage[mathletters]{ucs}
              \usepackage[utf8x]{inputenc}
          }
    \else
        \usepackage{fontspec}
        \usepackage{unicode-math}
    \fi

    \usepackage{fancyvrb} % verbatim replacement that allows latex
    \usepackage{grffile} % extends the file name processing of package graphics
                         % to support a larger range
    \makeatletter % fix for old versions of grffile with XeLaTeX
    \@ifpackagelater{grffile}{2019/11/01}
    {
      % Do nothing on new versions
    }
    {
      \def\Gread@@xetex#1{%
        \IfFileExists{"\Gin@base".bb}%
        {\Gread@eps{\Gin@base.bb}}%
        {\Gread@@xetex@aux#1}%
      }
    }
    \makeatother
    \usepackage[Export]{adjustbox} % Used to constrain images to a maximum size
    \adjustboxset{max size={0.9\linewidth}{0.9\paperheight}}

    % The hyperref package gives us a pdf with properly built
    % internal navigation ('pdf bookmarks' for the table of contents,
    % internal cross-reference links, web links for URLs, etc.)
    \usepackage{hyperref}
    % The default LaTeX title has an obnoxious amount of whitespace. By default,
    % titling removes some of it. It also provides customization options.
    \usepackage{titling}
    \usepackage{longtable} % longtable support required by pandoc >1.10
    \usepackage{booktabs}  % table support for pandoc > 1.12.2
    \usepackage{array}     % table support for pandoc >= 2.11.3
    \usepackage{calc}      % table minipage width calculation for pandoc >= 2.11.1
    \usepackage[inline]{enumitem} % IRkernel/repr support (it uses the enumerate* environment)
    \usepackage[normalem]{ulem} % ulem is needed to support strikethroughs (\sout)
                                % normalem makes italics be italics, not underlines
    \usepackage{soul}      % strikethrough (\st) support for pandoc >= 3.0.0
    \usepackage{mathrsfs}
    

    
    % Colors for the hyperref package
    \definecolor{urlcolor}{rgb}{0,.145,.698}
    \definecolor{linkcolor}{rgb}{.71,0.21,0.01}
    \definecolor{citecolor}{rgb}{.12,.54,.11}

    % ANSI colors
    \definecolor{ansi-black}{HTML}{3E424D}
    \definecolor{ansi-black-intense}{HTML}{282C36}
    \definecolor{ansi-red}{HTML}{E75C58}
    \definecolor{ansi-red-intense}{HTML}{B22B31}
    \definecolor{ansi-green}{HTML}{00A250}
    \definecolor{ansi-green-intense}{HTML}{007427}
    \definecolor{ansi-yellow}{HTML}{DDB62B}
    \definecolor{ansi-yellow-intense}{HTML}{B27D12}
    \definecolor{ansi-blue}{HTML}{208FFB}
    \definecolor{ansi-blue-intense}{HTML}{0065CA}
    \definecolor{ansi-magenta}{HTML}{D160C4}
    \definecolor{ansi-magenta-intense}{HTML}{A03196}
    \definecolor{ansi-cyan}{HTML}{60C6C8}
    \definecolor{ansi-cyan-intense}{HTML}{258F8F}
    \definecolor{ansi-white}{HTML}{C5C1B4}
    \definecolor{ansi-white-intense}{HTML}{A1A6B2}
    \definecolor{ansi-default-inverse-fg}{HTML}{FFFFFF}
    \definecolor{ansi-default-inverse-bg}{HTML}{000000}

    % common color for the border for error outputs.
    \definecolor{outerrorbackground}{HTML}{FFDFDF}

    % commands and environments needed by pandoc snippets
    % extracted from the output of `pandoc -s`
    \providecommand{\tightlist}{%
      \setlength{\itemsep}{0pt}\setlength{\parskip}{0pt}}
    \DefineVerbatimEnvironment{Highlighting}{Verbatim}{commandchars=\\\{\}}
    % Add ',fontsize=\small' for more characters per line
    \newenvironment{Shaded}{}{}
    \newcommand{\KeywordTok}[1]{\textcolor[rgb]{0.00,0.44,0.13}{\textbf{{#1}}}}
    \newcommand{\DataTypeTok}[1]{\textcolor[rgb]{0.56,0.13,0.00}{{#1}}}
    \newcommand{\DecValTok}[1]{\textcolor[rgb]{0.25,0.63,0.44}{{#1}}}
    \newcommand{\BaseNTok}[1]{\textcolor[rgb]{0.25,0.63,0.44}{{#1}}}
    \newcommand{\FloatTok}[1]{\textcolor[rgb]{0.25,0.63,0.44}{{#1}}}
    \newcommand{\CharTok}[1]{\textcolor[rgb]{0.25,0.44,0.63}{{#1}}}
    \newcommand{\StringTok}[1]{\textcolor[rgb]{0.25,0.44,0.63}{{#1}}}
    \newcommand{\CommentTok}[1]{\textcolor[rgb]{0.38,0.63,0.69}{\textit{{#1}}}}
    \newcommand{\OtherTok}[1]{\textcolor[rgb]{0.00,0.44,0.13}{{#1}}}
    \newcommand{\AlertTok}[1]{\textcolor[rgb]{1.00,0.00,0.00}{\textbf{{#1}}}}
    \newcommand{\FunctionTok}[1]{\textcolor[rgb]{0.02,0.16,0.49}{{#1}}}
    \newcommand{\RegionMarkerTok}[1]{{#1}}
    \newcommand{\ErrorTok}[1]{\textcolor[rgb]{1.00,0.00,0.00}{\textbf{{#1}}}}
    \newcommand{\NormalTok}[1]{{#1}}

    % Additional commands for more recent versions of Pandoc
    \newcommand{\ConstantTok}[1]{\textcolor[rgb]{0.53,0.00,0.00}{{#1}}}
    \newcommand{\SpecialCharTok}[1]{\textcolor[rgb]{0.25,0.44,0.63}{{#1}}}
    \newcommand{\VerbatimStringTok}[1]{\textcolor[rgb]{0.25,0.44,0.63}{{#1}}}
    \newcommand{\SpecialStringTok}[1]{\textcolor[rgb]{0.73,0.40,0.53}{{#1}}}
    \newcommand{\ImportTok}[1]{{#1}}
    \newcommand{\DocumentationTok}[1]{\textcolor[rgb]{0.73,0.13,0.13}{\textit{{#1}}}}
    \newcommand{\AnnotationTok}[1]{\textcolor[rgb]{0.38,0.63,0.69}{\textbf{\textit{{#1}}}}}
    \newcommand{\CommentVarTok}[1]{\textcolor[rgb]{0.38,0.63,0.69}{\textbf{\textit{{#1}}}}}
    \newcommand{\VariableTok}[1]{\textcolor[rgb]{0.10,0.09,0.49}{{#1}}}
    \newcommand{\ControlFlowTok}[1]{\textcolor[rgb]{0.00,0.44,0.13}{\textbf{{#1}}}}
    \newcommand{\OperatorTok}[1]{\textcolor[rgb]{0.40,0.40,0.40}{{#1}}}
    \newcommand{\BuiltInTok}[1]{{#1}}
    \newcommand{\ExtensionTok}[1]{{#1}}
    \newcommand{\PreprocessorTok}[1]{\textcolor[rgb]{0.74,0.48,0.00}{{#1}}}
    \newcommand{\AttributeTok}[1]{\textcolor[rgb]{0.49,0.56,0.16}{{#1}}}
    \newcommand{\InformationTok}[1]{\textcolor[rgb]{0.38,0.63,0.69}{\textbf{\textit{{#1}}}}}
    \newcommand{\WarningTok}[1]{\textcolor[rgb]{0.38,0.63,0.69}{\textbf{\textit{{#1}}}}}
    \makeatletter
    \newsavebox\pandoc@box
    \newcommand*\pandocbounded[1]{%
      \sbox\pandoc@box{#1}%
      % scaling factors for width and height
      \Gscale@div\@tempa\textheight{\dimexpr\ht\pandoc@box+\dp\pandoc@box\relax}%
      \Gscale@div\@tempb\linewidth{\wd\pandoc@box}%
      % select the smaller of both
      \ifdim\@tempb\p@<\@tempa\p@
        \let\@tempa\@tempb
      \fi
      % scaling accordingly (\@tempa < 1)
      \ifdim\@tempa\p@<\p@
        \scalebox{\@tempa}{\usebox\pandoc@box}%
      % scaling not needed, use as it is
      \else
        \usebox{\pandoc@box}%
      \fi
    }
    \makeatother

    % Define a nice break command that doesn't care if a line doesn't already
    % exist.
    \def\br{\hspace*{\fill} \\* }
    % Math Jax compatibility definitions
    \def\gt{>}
    \def\lt{<}
    \let\Oldtex\TeX
    \let\Oldlatex\LaTeX
    \renewcommand{\TeX}{\textrm{\Oldtex}}
    \renewcommand{\LaTeX}{\textrm{\Oldlatex}}
    % Document parameters
    % Document title
    \title{Exercise\_week\_03}
    
    
    
    
    
    
    
% Pygments definitions
\makeatletter
\def\PY@reset{\let\PY@it=\relax \let\PY@bf=\relax%
    \let\PY@ul=\relax \let\PY@tc=\relax%
    \let\PY@bc=\relax \let\PY@ff=\relax}
\def\PY@tok#1{\csname PY@tok@#1\endcsname}
\def\PY@toks#1+{\ifx\relax#1\empty\else%
    \PY@tok{#1}\expandafter\PY@toks\fi}
\def\PY@do#1{\PY@bc{\PY@tc{\PY@ul{%
    \PY@it{\PY@bf{\PY@ff{#1}}}}}}}
\def\PY#1#2{\PY@reset\PY@toks#1+\relax+\PY@do{#2}}

\@namedef{PY@tok@w}{\def\PY@tc##1{\textcolor[rgb]{0.73,0.73,0.73}{##1}}}
\@namedef{PY@tok@c}{\let\PY@it=\textit\def\PY@tc##1{\textcolor[rgb]{0.24,0.48,0.48}{##1}}}
\@namedef{PY@tok@cp}{\def\PY@tc##1{\textcolor[rgb]{0.61,0.40,0.00}{##1}}}
\@namedef{PY@tok@k}{\let\PY@bf=\textbf\def\PY@tc##1{\textcolor[rgb]{0.00,0.50,0.00}{##1}}}
\@namedef{PY@tok@kp}{\def\PY@tc##1{\textcolor[rgb]{0.00,0.50,0.00}{##1}}}
\@namedef{PY@tok@kt}{\def\PY@tc##1{\textcolor[rgb]{0.69,0.00,0.25}{##1}}}
\@namedef{PY@tok@o}{\def\PY@tc##1{\textcolor[rgb]{0.40,0.40,0.40}{##1}}}
\@namedef{PY@tok@ow}{\let\PY@bf=\textbf\def\PY@tc##1{\textcolor[rgb]{0.67,0.13,1.00}{##1}}}
\@namedef{PY@tok@nb}{\def\PY@tc##1{\textcolor[rgb]{0.00,0.50,0.00}{##1}}}
\@namedef{PY@tok@nf}{\def\PY@tc##1{\textcolor[rgb]{0.00,0.00,1.00}{##1}}}
\@namedef{PY@tok@nc}{\let\PY@bf=\textbf\def\PY@tc##1{\textcolor[rgb]{0.00,0.00,1.00}{##1}}}
\@namedef{PY@tok@nn}{\let\PY@bf=\textbf\def\PY@tc##1{\textcolor[rgb]{0.00,0.00,1.00}{##1}}}
\@namedef{PY@tok@ne}{\let\PY@bf=\textbf\def\PY@tc##1{\textcolor[rgb]{0.80,0.25,0.22}{##1}}}
\@namedef{PY@tok@nv}{\def\PY@tc##1{\textcolor[rgb]{0.10,0.09,0.49}{##1}}}
\@namedef{PY@tok@no}{\def\PY@tc##1{\textcolor[rgb]{0.53,0.00,0.00}{##1}}}
\@namedef{PY@tok@nl}{\def\PY@tc##1{\textcolor[rgb]{0.46,0.46,0.00}{##1}}}
\@namedef{PY@tok@ni}{\let\PY@bf=\textbf\def\PY@tc##1{\textcolor[rgb]{0.44,0.44,0.44}{##1}}}
\@namedef{PY@tok@na}{\def\PY@tc##1{\textcolor[rgb]{0.41,0.47,0.13}{##1}}}
\@namedef{PY@tok@nt}{\let\PY@bf=\textbf\def\PY@tc##1{\textcolor[rgb]{0.00,0.50,0.00}{##1}}}
\@namedef{PY@tok@nd}{\def\PY@tc##1{\textcolor[rgb]{0.67,0.13,1.00}{##1}}}
\@namedef{PY@tok@s}{\def\PY@tc##1{\textcolor[rgb]{0.73,0.13,0.13}{##1}}}
\@namedef{PY@tok@sd}{\let\PY@it=\textit\def\PY@tc##1{\textcolor[rgb]{0.73,0.13,0.13}{##1}}}
\@namedef{PY@tok@si}{\let\PY@bf=\textbf\def\PY@tc##1{\textcolor[rgb]{0.64,0.35,0.47}{##1}}}
\@namedef{PY@tok@se}{\let\PY@bf=\textbf\def\PY@tc##1{\textcolor[rgb]{0.67,0.36,0.12}{##1}}}
\@namedef{PY@tok@sr}{\def\PY@tc##1{\textcolor[rgb]{0.64,0.35,0.47}{##1}}}
\@namedef{PY@tok@ss}{\def\PY@tc##1{\textcolor[rgb]{0.10,0.09,0.49}{##1}}}
\@namedef{PY@tok@sx}{\def\PY@tc##1{\textcolor[rgb]{0.00,0.50,0.00}{##1}}}
\@namedef{PY@tok@m}{\def\PY@tc##1{\textcolor[rgb]{0.40,0.40,0.40}{##1}}}
\@namedef{PY@tok@gh}{\let\PY@bf=\textbf\def\PY@tc##1{\textcolor[rgb]{0.00,0.00,0.50}{##1}}}
\@namedef{PY@tok@gu}{\let\PY@bf=\textbf\def\PY@tc##1{\textcolor[rgb]{0.50,0.00,0.50}{##1}}}
\@namedef{PY@tok@gd}{\def\PY@tc##1{\textcolor[rgb]{0.63,0.00,0.00}{##1}}}
\@namedef{PY@tok@gi}{\def\PY@tc##1{\textcolor[rgb]{0.00,0.52,0.00}{##1}}}
\@namedef{PY@tok@gr}{\def\PY@tc##1{\textcolor[rgb]{0.89,0.00,0.00}{##1}}}
\@namedef{PY@tok@ge}{\let\PY@it=\textit}
\@namedef{PY@tok@gs}{\let\PY@bf=\textbf}
\@namedef{PY@tok@ges}{\let\PY@bf=\textbf\let\PY@it=\textit}
\@namedef{PY@tok@gp}{\let\PY@bf=\textbf\def\PY@tc##1{\textcolor[rgb]{0.00,0.00,0.50}{##1}}}
\@namedef{PY@tok@go}{\def\PY@tc##1{\textcolor[rgb]{0.44,0.44,0.44}{##1}}}
\@namedef{PY@tok@gt}{\def\PY@tc##1{\textcolor[rgb]{0.00,0.27,0.87}{##1}}}
\@namedef{PY@tok@err}{\def\PY@bc##1{{\setlength{\fboxsep}{\string -\fboxrule}\fcolorbox[rgb]{1.00,0.00,0.00}{1,1,1}{\strut ##1}}}}
\@namedef{PY@tok@kc}{\let\PY@bf=\textbf\def\PY@tc##1{\textcolor[rgb]{0.00,0.50,0.00}{##1}}}
\@namedef{PY@tok@kd}{\let\PY@bf=\textbf\def\PY@tc##1{\textcolor[rgb]{0.00,0.50,0.00}{##1}}}
\@namedef{PY@tok@kn}{\let\PY@bf=\textbf\def\PY@tc##1{\textcolor[rgb]{0.00,0.50,0.00}{##1}}}
\@namedef{PY@tok@kr}{\let\PY@bf=\textbf\def\PY@tc##1{\textcolor[rgb]{0.00,0.50,0.00}{##1}}}
\@namedef{PY@tok@bp}{\def\PY@tc##1{\textcolor[rgb]{0.00,0.50,0.00}{##1}}}
\@namedef{PY@tok@fm}{\def\PY@tc##1{\textcolor[rgb]{0.00,0.00,1.00}{##1}}}
\@namedef{PY@tok@vc}{\def\PY@tc##1{\textcolor[rgb]{0.10,0.09,0.49}{##1}}}
\@namedef{PY@tok@vg}{\def\PY@tc##1{\textcolor[rgb]{0.10,0.09,0.49}{##1}}}
\@namedef{PY@tok@vi}{\def\PY@tc##1{\textcolor[rgb]{0.10,0.09,0.49}{##1}}}
\@namedef{PY@tok@vm}{\def\PY@tc##1{\textcolor[rgb]{0.10,0.09,0.49}{##1}}}
\@namedef{PY@tok@sa}{\def\PY@tc##1{\textcolor[rgb]{0.73,0.13,0.13}{##1}}}
\@namedef{PY@tok@sb}{\def\PY@tc##1{\textcolor[rgb]{0.73,0.13,0.13}{##1}}}
\@namedef{PY@tok@sc}{\def\PY@tc##1{\textcolor[rgb]{0.73,0.13,0.13}{##1}}}
\@namedef{PY@tok@dl}{\def\PY@tc##1{\textcolor[rgb]{0.73,0.13,0.13}{##1}}}
\@namedef{PY@tok@s2}{\def\PY@tc##1{\textcolor[rgb]{0.73,0.13,0.13}{##1}}}
\@namedef{PY@tok@sh}{\def\PY@tc##1{\textcolor[rgb]{0.73,0.13,0.13}{##1}}}
\@namedef{PY@tok@s1}{\def\PY@tc##1{\textcolor[rgb]{0.73,0.13,0.13}{##1}}}
\@namedef{PY@tok@mb}{\def\PY@tc##1{\textcolor[rgb]{0.40,0.40,0.40}{##1}}}
\@namedef{PY@tok@mf}{\def\PY@tc##1{\textcolor[rgb]{0.40,0.40,0.40}{##1}}}
\@namedef{PY@tok@mh}{\def\PY@tc##1{\textcolor[rgb]{0.40,0.40,0.40}{##1}}}
\@namedef{PY@tok@mi}{\def\PY@tc##1{\textcolor[rgb]{0.40,0.40,0.40}{##1}}}
\@namedef{PY@tok@il}{\def\PY@tc##1{\textcolor[rgb]{0.40,0.40,0.40}{##1}}}
\@namedef{PY@tok@mo}{\def\PY@tc##1{\textcolor[rgb]{0.40,0.40,0.40}{##1}}}
\@namedef{PY@tok@ch}{\let\PY@it=\textit\def\PY@tc##1{\textcolor[rgb]{0.24,0.48,0.48}{##1}}}
\@namedef{PY@tok@cm}{\let\PY@it=\textit\def\PY@tc##1{\textcolor[rgb]{0.24,0.48,0.48}{##1}}}
\@namedef{PY@tok@cpf}{\let\PY@it=\textit\def\PY@tc##1{\textcolor[rgb]{0.24,0.48,0.48}{##1}}}
\@namedef{PY@tok@c1}{\let\PY@it=\textit\def\PY@tc##1{\textcolor[rgb]{0.24,0.48,0.48}{##1}}}
\@namedef{PY@tok@cs}{\let\PY@it=\textit\def\PY@tc##1{\textcolor[rgb]{0.24,0.48,0.48}{##1}}}

\def\PYZbs{\char`\\}
\def\PYZus{\char`\_}
\def\PYZob{\char`\{}
\def\PYZcb{\char`\}}
\def\PYZca{\char`\^}
\def\PYZam{\char`\&}
\def\PYZlt{\char`\<}
\def\PYZgt{\char`\>}
\def\PYZsh{\char`\#}
\def\PYZpc{\char`\%}
\def\PYZdl{\char`\$}
\def\PYZhy{\char`\-}
\def\PYZsq{\char`\'}
\def\PYZdq{\char`\"}
\def\PYZti{\char`\~}
% for compatibility with earlier versions
\def\PYZat{@}
\def\PYZlb{[}
\def\PYZrb{]}
\makeatother


    % For linebreaks inside Verbatim environment from package fancyvrb.
    \makeatletter
        \newbox\Wrappedcontinuationbox
        \newbox\Wrappedvisiblespacebox
        \newcommand*\Wrappedvisiblespace {\textcolor{red}{\textvisiblespace}}
        \newcommand*\Wrappedcontinuationsymbol {\textcolor{red}{\llap{\tiny$\m@th\hookrightarrow$}}}
        \newcommand*\Wrappedcontinuationindent {3ex }
        \newcommand*\Wrappedafterbreak {\kern\Wrappedcontinuationindent\copy\Wrappedcontinuationbox}
        % Take advantage of the already applied Pygments mark-up to insert
        % potential linebreaks for TeX processing.
        %        {, <, #, %, $, ' and ": go to next line.
        %        _, }, ^, &, >, - and ~: stay at end of broken line.
        % Use of \textquotesingle for straight quote.
        \newcommand*\Wrappedbreaksatspecials {%
            \def\PYGZus{\discretionary{\char`\_}{\Wrappedafterbreak}{\char`\_}}%
            \def\PYGZob{\discretionary{}{\Wrappedafterbreak\char`\{}{\char`\{}}%
            \def\PYGZcb{\discretionary{\char`\}}{\Wrappedafterbreak}{\char`\}}}%
            \def\PYGZca{\discretionary{\char`\^}{\Wrappedafterbreak}{\char`\^}}%
            \def\PYGZam{\discretionary{\char`\&}{\Wrappedafterbreak}{\char`\&}}%
            \def\PYGZlt{\discretionary{}{\Wrappedafterbreak\char`\<}{\char`\<}}%
            \def\PYGZgt{\discretionary{\char`\>}{\Wrappedafterbreak}{\char`\>}}%
            \def\PYGZsh{\discretionary{}{\Wrappedafterbreak\char`\#}{\char`\#}}%
            \def\PYGZpc{\discretionary{}{\Wrappedafterbreak\char`\%}{\char`\%}}%
            \def\PYGZdl{\discretionary{}{\Wrappedafterbreak\char`\$}{\char`\$}}%
            \def\PYGZhy{\discretionary{\char`\-}{\Wrappedafterbreak}{\char`\-}}%
            \def\PYGZsq{\discretionary{}{\Wrappedafterbreak\textquotesingle}{\textquotesingle}}%
            \def\PYGZdq{\discretionary{}{\Wrappedafterbreak\char`\"}{\char`\"}}%
            \def\PYGZti{\discretionary{\char`\~}{\Wrappedafterbreak}{\char`\~}}%
        }
        % Some characters . , ; ? ! / are not pygmentized.
        % This macro makes them "active" and they will insert potential linebreaks
        \newcommand*\Wrappedbreaksatpunct {%
            \lccode`\~`\.\lowercase{\def~}{\discretionary{\hbox{\char`\.}}{\Wrappedafterbreak}{\hbox{\char`\.}}}%
            \lccode`\~`\,\lowercase{\def~}{\discretionary{\hbox{\char`\,}}{\Wrappedafterbreak}{\hbox{\char`\,}}}%
            \lccode`\~`\;\lowercase{\def~}{\discretionary{\hbox{\char`\;}}{\Wrappedafterbreak}{\hbox{\char`\;}}}%
            \lccode`\~`\:\lowercase{\def~}{\discretionary{\hbox{\char`\:}}{\Wrappedafterbreak}{\hbox{\char`\:}}}%
            \lccode`\~`\?\lowercase{\def~}{\discretionary{\hbox{\char`\?}}{\Wrappedafterbreak}{\hbox{\char`\?}}}%
            \lccode`\~`\!\lowercase{\def~}{\discretionary{\hbox{\char`\!}}{\Wrappedafterbreak}{\hbox{\char`\!}}}%
            \lccode`\~`\/\lowercase{\def~}{\discretionary{\hbox{\char`\/}}{\Wrappedafterbreak}{\hbox{\char`\/}}}%
            \catcode`\.\active
            \catcode`\,\active
            \catcode`\;\active
            \catcode`\:\active
            \catcode`\?\active
            \catcode`\!\active
            \catcode`\/\active
            \lccode`\~`\~
        }
    \makeatother

    \let\OriginalVerbatim=\Verbatim
    \makeatletter
    \renewcommand{\Verbatim}[1][1]{%
        %\parskip\z@skip
        \sbox\Wrappedcontinuationbox {\Wrappedcontinuationsymbol}%
        \sbox\Wrappedvisiblespacebox {\FV@SetupFont\Wrappedvisiblespace}%
        \def\FancyVerbFormatLine ##1{\hsize\linewidth
            \vtop{\raggedright\hyphenpenalty\z@\exhyphenpenalty\z@
                \doublehyphendemerits\z@\finalhyphendemerits\z@
                \strut ##1\strut}%
        }%
        % If the linebreak is at a space, the latter will be displayed as visible
        % space at end of first line, and a continuation symbol starts next line.
        % Stretch/shrink are however usually zero for typewriter font.
        \def\FV@Space {%
            \nobreak\hskip\z@ plus\fontdimen3\font minus\fontdimen4\font
            \discretionary{\copy\Wrappedvisiblespacebox}{\Wrappedafterbreak}
            {\kern\fontdimen2\font}%
        }%

        % Allow breaks at special characters using \PYG... macros.
        \Wrappedbreaksatspecials
        % Breaks at punctuation characters . , ; ? ! and / need catcode=\active
        \OriginalVerbatim[#1,codes*=\Wrappedbreaksatpunct]%
    }
    \makeatother

    % Exact colors from NB
    \definecolor{incolor}{HTML}{303F9F}
    \definecolor{outcolor}{HTML}{D84315}
    \definecolor{cellborder}{HTML}{CFCFCF}
    \definecolor{cellbackground}{HTML}{F7F7F7}

    % prompt
    \makeatletter
    \newcommand{\boxspacing}{\kern\kvtcb@left@rule\kern\kvtcb@boxsep}
    \makeatother
    \newcommand{\prompt}[4]{
        {\ttfamily\llap{{\color{#2}[#3]:\hspace{3pt}#4}}\vspace{-\baselineskip}}
    }
    

    
    % Prevent overflowing lines due to hard-to-break entities
    \sloppy
    % Setup hyperref package
    \hypersetup{
      breaklinks=true,  % so long urls are correctly broken across lines
      colorlinks=true,
      urlcolor=urlcolor,
      linkcolor=linkcolor,
      citecolor=citecolor,
      }
    % Slightly bigger margins than the latex defaults
    
    \geometry{verbose,tmargin=1in,bmargin=1in,lmargin=1in,rmargin=1in}
    
    

\begin{document}
    
    \maketitle
    
    

    
    \section{Statistical Methods in Astrophysics
Exercises}\label{statistical-methods-in-astrophysics-exercises}

\subsection{Week 03: Probability Distribution
Functions}\label{week-03-probability-distribution-functions}

\subsubsection{Personal Information}\label{personal-information}

\textbf{Name:} 刘乐融

\textbf{Student ID:} 2024011182

    \subsubsection{Exercise 1: Steady Astronomical
Source}\label{exercise-1-steady-astronomical-source}

A steady astronomical source has flux
\(\phi = 100\,{\rm photon}\,{\rm s}^{-1}\). It is observed for a total
exposure time of \(T = 60\,{\rm s}\). Assume that the detection
efficiency is 100\%.

\paragraph{Question 1.1: Distribution of photon inter-arrival
intervals}\label{question-1.1-distribution-of-photon-inter-arrival-intervals}

\textbf{Tasks:} 1. Simulate the arrival times of photons from this
source. 2. Plot the distribution of time intervals between consecutive
photon arrivals with appropriate axes scales. 3. Overlay the theoretical
distribution of time intervals for a Poisson process with the same rate,
with the formula derived in the lecture slides:
\(p(\tau) = \phi e^{-\phi \tau}\), where \(\tau\) is the time interval
between consecutive arrivals.

    \begin{tcolorbox}[breakable, size=fbox, boxrule=1pt, pad at break*=1mm,colback=cellbackground, colframe=cellborder]
\prompt{In}{incolor}{2}{\boxspacing}
\begin{Verbatim}[commandchars=\\\{\}]
\PY{c+c1}{\PYZsh{} NOTE: Run this cell in the first place}
\PY{c+c1}{\PYZsh{} Load packages for numerical calculations and plotting}

\PY{k+kn}{import}\PY{+w}{ }\PY{n+nn}{numpy}\PY{+w}{ }\PY{k}{as}\PY{+w}{ }\PY{n+nn}{np}
\PY{k+kn}{import}\PY{+w}{ }\PY{n+nn}{matplotlib}\PY{n+nn}{.}\PY{n+nn}{pyplot}\PY{+w}{ }\PY{k}{as}\PY{+w}{ }\PY{n+nn}{plt}
\PY{c+c1}{\PYZsh{} Enable inline plotting in Jupyter notebooks}
\PY{n}{plt}\PY{o}{.}\PY{n}{rcParams}\PY{p}{[}\PY{l+s+s1}{\PYZsq{}}\PY{l+s+s1}{figure.figsize}\PY{l+s+s1}{\PYZsq{}}\PY{p}{]} \PY{o}{=} \PY{p}{(}\PY{l+m+mf}{9.6}\PY{p}{,} \PY{l+m+mf}{5.4}\PY{p}{)}
\PY{n}{plt}\PY{o}{.}\PY{n}{rcParams}\PY{p}{[}\PY{l+s+s1}{\PYZsq{}}\PY{l+s+s1}{figure.dpi}\PY{l+s+s1}{\PYZsq{}}\PY{p}{]} \PY{o}{=} \PY{l+m+mi}{300}
\PY{o}{\PYZpc{}}\PY{k}{matplotlib} inline
\PY{n}{plt}\PY{o}{.}\PY{n}{rc}\PY{p}{(}\PY{l+s+s1}{\PYZsq{}}\PY{l+s+s1}{text}\PY{l+s+s1}{\PYZsq{}}\PY{p}{,} \PY{n}{usetex} \PY{o}{=} \PY{k+kc}{True}\PY{p}{)}
\PY{n}{plt}\PY{o}{.}\PY{n}{rc}\PY{p}{(}\PY{l+s+s1}{\PYZsq{}}\PY{l+s+s1}{font}\PY{l+s+s1}{\PYZsq{}}\PY{p}{,} \PY{n}{family} \PY{o}{=} \PY{l+s+s1}{\PYZsq{}}\PY{l+s+s1}{serif}\PY{l+s+s1}{\PYZsq{}}\PY{p}{)}
\PY{c+c1}{\PYZsh{} 由于设置了图片的高 dpi 和字体,所以编译速度变慢.}
\end{Verbatim}
\end{tcolorbox}

    \begin{tcolorbox}[breakable, size=fbox, boxrule=1pt, pad at break*=1mm,colback=cellbackground, colframe=cellborder]
\prompt{In}{incolor}{3}{\boxspacing}
\begin{Verbatim}[commandchars=\\\{\}]
\PY{k}{def}\PY{+w}{ }\PY{n+nf}{photon\PYZus{}arrivals}\PY{p}{(}\PY{n}{flux}\PY{p}{,} \PY{n}{duration}\PY{p}{,} \PY{n}{seed} \PY{o}{=} \PY{l+m+mi}{42}\PY{p}{)}\PY{p}{:}
\PY{+w}{    }\PY{l+s+sd}{\PYZdq{}\PYZdq{}\PYZdq{}Simulate photon arrival times for a steady source.}
\PY{l+s+sd}{    Args:}
\PY{l+s+sd}{        flux (float): Photon flux in photons per second.}
\PY{l+s+sd}{        duration (float): Total observation time in seconds.}
\PY{l+s+sd}{        seed (int): Random seed for reproducibility.}
\PY{l+s+sd}{    Returns:}
\PY{l+s+sd}{        Array of photon arrival times in seconds.}
\PY{l+s+sd}{    \PYZdq{}\PYZdq{}\PYZdq{}}
    \PY{n}{rng} \PY{o}{=} \PY{n}{np}\PY{o}{.}\PY{n}{random}\PY{o}{.}\PY{n}{default\PYZus{}rng}\PY{p}{(}\PY{n}{seed}\PY{p}{)}
    \PY{n}{count} \PY{o}{=} \PY{n}{rng}\PY{o}{.}\PY{n}{poisson}\PY{p}{(}\PY{n}{flux} \PY{o}{*} \PY{n}{duration}\PY{p}{)} \PY{c+c1}{\PYZsh{} Sample total number of photons}
    \PY{n}{arrivals} \PY{o}{=} \PY{n}{rng}\PY{o}{.}\PY{n}{uniform}\PY{p}{(}\PY{l+m+mi}{0}\PY{p}{,} \PY{n}{duration}\PY{p}{,} \PY{n}{size} \PY{o}{=} \PY{n}{count}\PY{p}{)} \PY{c+c1}{\PYZsh{} Sample arrival times}
    \PY{n}{arrivals}\PY{o}{.}\PY{n}{sort}\PY{p}{(}\PY{p}{)}
    \PY{k}{return} \PY{n}{arrivals}
\end{Verbatim}
\end{tcolorbox}

    \begin{tcolorbox}[breakable, size=fbox, boxrule=1pt, pad at break*=1mm,colback=cellbackground, colframe=cellborder]
\prompt{In}{incolor}{4}{\boxspacing}
\begin{Verbatim}[commandchars=\\\{\}]
\PY{c+c1}{\PYZsh{} Set the random seed}
\PY{n}{seed} \PY{o}{=} \PY{l+m+mi}{2024011182} \PY{c+c1}{\PYZsh{} NOTE: EDIT HERE to insert your student ID as the seed}

\PY{n}{flux} \PY{o}{=} \PY{l+m+mi}{100} \PY{c+c1}{\PYZsh{} photon/s}
\PY{n}{duration} \PY{o}{=} \PY{l+m+mi}{60} \PY{c+c1}{\PYZsh{} s}
\PY{n}{arrivals} \PY{o}{=} \PY{n}{photon\PYZus{}arrivals}\PY{p}{(}\PY{n}{flux}\PY{p}{,} \PY{n}{duration}\PY{p}{,} \PY{n}{seed} \PY{o}{=} \PY{n}{seed}\PY{p}{)}
\PY{n}{intervals} \PY{o}{=} \PY{n}{np}\PY{o}{.}\PY{n}{diff}\PY{p}{(}\PY{n}{arrivals}\PY{p}{)}

\PY{c+c1}{\PYZsh{} NOTE: EDIT BELOW to plot the histogram of intervals and overlay the theoretical distribution}
\PY{c+c1}{\PYZsh{} Plot histogram of inter\PYZhy{}arrival intervals and overlay the theoretical exponential distribution}
\PY{n}{plt}\PY{o}{.}\PY{n}{hist}\PY{p}{(}
    \PY{n}{intervals}\PY{p}{,} 
    \PY{n}{bins} \PY{o}{=} \PY{l+m+mi}{50}\PY{p}{,} 
    \PY{n}{density} \PY{o}{=} \PY{k+kc}{True}\PY{p}{,} 
    \PY{n}{label} \PY{o}{=} \PY{l+s+s1}{\PYZsq{}}\PY{l+s+s1}{Simulated}\PY{l+s+s1}{\PYZsq{}}\PY{p}{,}
\PY{p}{)}

\PY{n}{tau} \PY{o}{=} \PY{n}{np}\PY{o}{.}\PY{n}{linspace}\PY{p}{(}\PY{l+m+mi}{0}\PY{p}{,} \PY{n}{intervals}\PY{o}{.}\PY{n}{max}\PY{p}{(}\PY{p}{)}\PY{p}{,} \PY{l+m+mi}{500}\PY{p}{)}
\PY{n}{p\PYZus{}tau} \PY{o}{=} \PY{n}{flux} \PY{o}{*} \PY{n}{np}\PY{o}{.}\PY{n}{exp}\PY{p}{(}\PY{o}{\PYZhy{}}\PY{n}{flux} \PY{o}{*} \PY{n}{tau}\PY{p}{)} \PY{c+c1}{\PYZsh{} p(\PYZbs{}tau) = \PYZbs{}phi * \PYZbs{}exp(\PYZhy{}\PYZbs{}phi * \PYZbs{}tau)}
\PY{n}{plt}\PY{o}{.}\PY{n}{plot}\PY{p}{(}
    \PY{n}{tau}\PY{p}{,} 
    \PY{n}{p\PYZus{}tau}\PY{p}{,}
    \PY{l+s+s1}{\PYZsq{}}\PY{l+s+s1}{r\PYZhy{}}\PY{l+s+s1}{\PYZsq{}}\PY{p}{,} 
    \PY{n}{label} \PY{o}{=} \PY{l+s+sa}{r}\PY{l+s+s1}{\PYZsq{}}\PY{l+s+s1}{Theory: \PYZdl{}}\PY{l+s+s1}{\PYZbs{}}\PY{l+s+s1}{phi e\PYZca{}}\PY{l+s+s1}{\PYZob{}}\PY{l+s+s1}{\PYZhy{}}\PY{l+s+s1}{\PYZbs{}}\PY{l+s+s1}{phi }\PY{l+s+s1}{\PYZbs{}}\PY{l+s+s1}{tau\PYZcb{}\PYZdl{}}\PY{l+s+s1}{\PYZsq{}}\PY{p}{,}
\PY{p}{)}
\PY{n}{plt}\PY{o}{.}\PY{n}{yscale}\PY{p}{(}\PY{l+s+s1}{\PYZsq{}}\PY{l+s+s1}{log}\PY{l+s+s1}{\PYZsq{}}\PY{p}{)}
\PY{n}{plt}\PY{o}{.}\PY{n}{xlabel}\PY{p}{(}\PY{l+s+s1}{\PYZsq{}}\PY{l+s+s1}{Inter\PYZhy{}arrival time (s)}\PY{l+s+s1}{\PYZsq{}}\PY{p}{)}
\PY{n}{plt}\PY{o}{.}\PY{n}{ylabel}\PY{p}{(}\PY{l+s+s1}{\PYZsq{}}\PY{l+s+s1}{Probability density}\PY{l+s+s1}{\PYZsq{}}\PY{p}{)}
\PY{n}{plt}\PY{o}{.}\PY{n}{legend}\PY{p}{(}\PY{p}{)}
\PY{n}{plt}\PY{o}{.}\PY{n}{show}\PY{p}{(}\PY{p}{)}
\end{Verbatim}
\end{tcolorbox}

    \begin{center}
    \adjustimage{max size={0.9\linewidth}{0.9\paperheight}}{Exercise_week_03_files/Exercise_week_03_4_0.png}
    \end{center}
    { \hspace*{\fill} \\}
    
    \paragraph{Question 1.2: Poisson clumps with short
exposures}\label{question-1.2-poisson-clumps-with-short-exposures}

Split the total exposure into short frames of
\(t_{\rm fr} = 10\,{\rm ms}\).

\textbf{Tasks:} 1. For the simulation in Question 1.1, bin the arrivals
into frames of length \(t_{\rm fr} = 10\,{\rm ms}\). 2. Compute the
fraction of frames with at least 5 photons (i.e., five times the
per-frame expectation) for each run, and report the mean fraction over
the 100 runs.

    \begin{tcolorbox}[breakable, size=fbox, boxrule=1pt, pad at break*=1mm,colback=cellbackground, colframe=cellborder]
\prompt{In}{incolor}{5}{\boxspacing}
\begin{Verbatim}[commandchars=\\\{\}]
\PY{n}{t\PYZus{}fr} \PY{o}{=} \PY{l+m+mf}{0.01} \PY{c+c1}{\PYZsh{} s}
\PY{n}{bin\PYZus{}edges} \PY{o}{=} \PY{n}{np}\PY{o}{.}\PY{n}{arange}\PY{p}{(}\PY{l+m+mi}{0}\PY{p}{,} \PY{n}{duration} \PY{o}{+} \PY{n}{t\PYZus{}fr}\PY{p}{,} \PY{n}{t\PYZus{}fr}\PY{p}{)}
\PY{n}{counts}\PY{p}{,} \PY{n}{\PYZus{}} \PY{o}{=} \PY{n}{np}\PY{o}{.}\PY{n}{histogram}\PY{p}{(}\PY{n}{arrivals}\PY{p}{,} \PY{n}{bins} \PY{o}{=} \PY{n}{bin\PYZus{}edges}\PY{p}{)}
\PY{c+c1}{\PYZsh{} compute the fraction of exposures (frames) with at least 5 photons}
\PY{n}{fraction} \PY{o}{=} \PY{n}{np}\PY{o}{.}\PY{n}{count\PYZus{}nonzero}\PY{p}{(}\PY{n}{counts} \PY{o}{\PYZgt{}}\PY{o}{=} \PY{l+m+mi}{5}\PY{p}{)} \PY{o}{/} \PY{n}{counts}\PY{o}{.}\PY{n}{size}

\PY{n+nb}{print}\PY{p}{(}\PY{l+s+sa}{f}\PY{l+s+s2}{\PYZdq{}}\PY{l+s+s2}{Mean fraction of exposures with at least 5 photons: }\PY{l+s+si}{\PYZob{}}\PY{n}{fraction}\PY{l+s+si}{:}\PY{l+s+s2}{.2\PYZpc{}}\PY{l+s+si}{\PYZcb{}}\PY{l+s+s2}{\PYZdq{}}\PY{p}{)}
\end{Verbatim}
\end{tcolorbox}

    \begin{Verbatim}[commandchars=\\\{\}]
Mean fraction of exposures with at least 5 photons: 0.32\%
    \end{Verbatim}

    \paragraph{\texorpdfstring{Question 1.3: Waiting time to the \(k\)-th
photon}{Question 1.3: Waiting time to the k-th photon}}\label{question-1.3-waiting-time-to-the-k-th-photon}

Let \(t_k\) be the waiting time from \(t=0\) to the arrival of the
\(k\)-th photon. The probability distribution function of \(t_k\) is
given by the Gamma (Erlang) distribution:

\[p(t_k) = \frac{\phi^k t_k^{k-1} e^{-\phi t_k}}{(k-1)!}\]

\textbf{Tasks:} 1. Using the same simulation results from Question 1.1,
plot the distribution of \(t_5\), i.e., the waiting time to the 5th
photon. 2. Overlay the theoretical distribution of \(t_5\) using the
formula above.

    \begin{tcolorbox}[breakable, size=fbox, boxrule=1pt, pad at break*=1mm,colback=cellbackground, colframe=cellborder]
\prompt{In}{incolor}{6}{\boxspacing}
\begin{Verbatim}[commandchars=\\\{\}]
\PY{c+c1}{\PYZsh{} For each photon, compute the waiting time to the 5th photon after it}
\PY{n}{t5} \PY{o}{=} \PY{n}{np}\PY{o}{.}\PY{n}{convolve}\PY{p}{(}\PY{n}{intervals}\PY{p}{,} \PY{n}{np}\PY{o}{.}\PY{n}{ones}\PY{p}{(}\PY{l+m+mi}{5}\PY{p}{)}\PY{p}{,} \PY{n}{mode} \PY{o}{=} \PY{l+s+s1}{\PYZsq{}}\PY{l+s+s1}{valid}\PY{l+s+s1}{\PYZsq{}}\PY{p}{)}

\PY{c+c1}{\PYZsh{} NOTE: EDIT BELOW to plot the histogram of t5 and overlay the theoretical distribution}
\PY{n}{plt}\PY{o}{.}\PY{n}{hist}\PY{p}{(}
    \PY{n}{t5}\PY{p}{,}
    \PY{n}{bins} \PY{o}{=} \PY{l+m+mi}{50}\PY{p}{,}
    \PY{n}{density} \PY{o}{=} \PY{k+kc}{True}\PY{p}{,}
    \PY{n}{label} \PY{o}{=} \PY{l+s+s1}{\PYZsq{}}\PY{l+s+s1}{Simulated}\PY{l+s+s1}{\PYZsq{}}\PY{p}{,}
\PY{p}{)}

\PY{n}{tau\PYZus{}5} \PY{o}{=} \PY{n}{np}\PY{o}{.}\PY{n}{linspace}\PY{p}{(}\PY{l+m+mi}{0}\PY{p}{,} \PY{n}{t5}\PY{o}{.}\PY{n}{max}\PY{p}{(}\PY{p}{)}\PY{p}{,} \PY{l+m+mi}{500}\PY{p}{)}
\PY{n}{p\PYZus{}tau\PYZus{}5} \PY{o}{=} \PY{n}{flux}\PY{o}{*}\PY{o}{*}\PY{l+m+mi}{5} \PY{o}{*} \PY{n}{tau\PYZus{}5}\PY{o}{*}\PY{o}{*}\PY{l+m+mi}{4} \PY{o}{*} \PY{n}{np}\PY{o}{.}\PY{n}{exp}\PY{p}{(}\PY{o}{\PYZhy{}}\PY{n}{flux} \PY{o}{*} \PY{n}{tau\PYZus{}5}\PY{p}{)} \PY{o}{/} \PY{l+m+mi}{24} \PY{c+c1}{\PYZsh{} \PYZbs{}phi\PYZca{}k * t\PYZus{}k\PYZca{}\PYZob{}k\PYZhy{}1\PYZcb{} * e\PYZca{}\PYZob{}\PYZhy{}\PYZbs{}phi t\PYZus{}k\PYZcb{}/(k\PYZhy{}1)!}
\PY{n}{plt}\PY{o}{.}\PY{n}{plot}\PY{p}{(}
    \PY{n}{tau\PYZus{}5}\PY{p}{,}
    \PY{n}{p\PYZus{}tau\PYZus{}5}\PY{p}{,}
    \PY{l+s+s1}{\PYZsq{}}\PY{l+s+s1}{r\PYZhy{}}\PY{l+s+s1}{\PYZsq{}}\PY{p}{,}
    \PY{n}{label} \PY{o}{=} \PY{l+s+sa}{r}\PY{l+s+s1}{\PYZsq{}}\PY{l+s+s1}{Theory: \PYZdl{}}\PY{l+s+s1}{\PYZbs{}}\PY{l+s+s1}{frac}\PY{l+s+s1}{\PYZob{}}\PY{l+s+s1}{\PYZbs{}}\PY{l+s+s1}{phi\PYZca{}k t\PYZus{}k\PYZca{}}\PY{l+s+s1}{\PYZob{}}\PY{l+s+s1}{k\PYZhy{}1\PYZcb{} e\PYZca{}}\PY{l+s+s1}{\PYZob{}}\PY{l+s+s1}{\PYZhy{}}\PY{l+s+s1}{\PYZbs{}}\PY{l+s+s1}{phi t\PYZus{}k\PYZcb{}\PYZcb{}}\PY{l+s+s1}{\PYZob{}}\PY{l+s+s1}{(k\PYZhy{}1)!\PYZcb{}\PYZdl{}}\PY{l+s+s1}{\PYZsq{}}
\PY{p}{)}

\PY{n}{plt}\PY{o}{.}\PY{n}{xlabel}\PY{p}{(}\PY{l+s+s1}{\PYZsq{}}\PY{l+s+s1}{Waiting time for the \PYZdl{}5\PYZdl{}\PYZhy{}th photon (s)}\PY{l+s+s1}{\PYZsq{}}\PY{p}{)}
\PY{n}{plt}\PY{o}{.}\PY{n}{ylabel}\PY{p}{(}\PY{l+s+s1}{\PYZsq{}}\PY{l+s+s1}{Probability density}\PY{l+s+s1}{\PYZsq{}}\PY{p}{)}
\PY{n}{plt}\PY{o}{.}\PY{n}{legend}\PY{p}{(}\PY{p}{)}
\PY{n}{plt}\PY{o}{.}\PY{n}{show}\PY{p}{(}\PY{p}{)}
\end{Verbatim}
\end{tcolorbox}

    \begin{center}
    \adjustimage{max size={0.9\linewidth}{0.9\paperheight}}{Exercise_week_03_files/Exercise_week_03_8_0.png}
    \end{center}
    { \hspace*{\fill} \\}
    
    \subsubsection{Exercise 2: Central Limit Theorem
(CLT)}\label{exercise-2-central-limit-theorem-clt}

Let \(\{ X_i \}_{i=1}^n\) be a sequence of independent random variables
drawn from the same distribution with mean \(\mu\) and finite variance
\(\sigma^2\). The Central Limit Theorem states that the distribution of

\[Z_n = \frac{\sum_{i=1}^n X_i - n\mu}{\sigma \sqrt{n}}\]

converges to the standard normal distribution (Gaussian distribution
with mean 0 and unit variance) as \(n \to \infty\).

\paragraph{Question 2.1: Source
brightness}\label{question-2.1-source-brightness}

The same steady astronomical source with flux
\(\phi = 100\,{\rm photon}\,{\rm s}^{-1}\) as in Exercise 1 is observed
with short frames of \(t_{\rm fr} = 10\,{\rm ms}\). Assume a detection
efficiency of 100\%.

The measured brightness (flux) of the source with a total exposure time
with \(N_{\rm fr}\) frames is given by the total number of detected
photons divided by the total exposure time:

\[\tilde{\phi} = \frac{N_{\rm detected}}{N_{\rm fr} \cdot t_{\rm fr}}\]

\textbf{Tasks:} 1. Simulate the photon counts in 256,000 frames. Plot
the distribution of the measured brightness \(\tilde{\phi}\) with
single-frame exposures. Overlay the theoretical distribution of
\(\tilde{\phi}\). 2. For multiple-frame exposures with
\(N_{\rm fr} = 1, 4, 16, 64, 256, 1024\), compute the measured
brightness \(\tilde{\phi}\) for each exposure. Plot the distribution of
\(\tilde{\phi}\) for each \(N_{\rm fr}\). 3. Overlay a theoretical
Gaussian distribution on each histogram.

\textbf{Hint:} From the CLT, derive what the mean and standard deviation
of the Gaussian distribution are, and how they depend on \(N_{\rm fr}\).

    \begin{tcolorbox}[breakable, size=fbox, boxrule=1pt, pad at break*=1mm,colback=cellbackground, colframe=cellborder]
\prompt{In}{incolor}{7}{\boxspacing}
\begin{Verbatim}[commandchars=\\\{\}]
\PY{c+c1}{\PYZsh{} Set the random seed}
\PY{n}{seed} \PY{o}{=} \PY{l+m+mi}{2024011182} \PY{c+c1}{\PYZsh{} NOTE: EDIT HERE to insert your student ID as the seed}

\PY{n}{flux} \PY{o}{=} \PY{l+m+mi}{100} \PY{c+c1}{\PYZsh{} photon/s}
\PY{n}{t\PYZus{}fr} \PY{o}{=} \PY{l+m+mf}{0.01} \PY{c+c1}{\PYZsh{} frame time: 10 ms}
\PY{n}{nsamples} \PY{o}{=} \PY{l+m+mi}{256000} \PY{c+c1}{\PYZsh{} number of frames to simulate}

\PY{c+c1}{\PYZsh{} Simulate photon counts in each frame}
\PY{n}{rng} \PY{o}{=} \PY{n}{np}\PY{o}{.}\PY{n}{random}\PY{o}{.}\PY{n}{default\PYZus{}rng}\PY{p}{(}\PY{n}{seed}\PY{p}{)}
\PY{n}{num\PYZus{}photons} \PY{o}{=} \PY{n}{rng}\PY{o}{.}\PY{n}{poisson}\PY{p}{(}\PY{n}{flux} \PY{o}{*} \PY{n}{t\PYZus{}fr}\PY{p}{,} \PY{n}{size} \PY{o}{=} \PY{n}{nsamples}\PY{p}{)}
\end{Verbatim}
\end{tcolorbox}

    \begin{tcolorbox}[breakable, size=fbox, boxrule=1pt, pad at break*=1mm,colback=cellbackground, colframe=cellborder]
\prompt{In}{incolor}{8}{\boxspacing}
\begin{Verbatim}[commandchars=\\\{\}]
\PY{c+c1}{\PYZsh{} NOTE: EDIT BELOW to plot the histogram of num\PYZus{}photons and overlay the theoretical Poisson distribution}
\PY{c+c1}{\PYZsh{} Use integer\PYZhy{}centered bins for counts and overlay the Poisson PMF for k = 0..kmax}
\PY{n}{plt}\PY{o}{.}\PY{n}{hist}\PY{p}{(}
    \PY{n}{num\PYZus{}photons}\PY{p}{,}
    \PY{n}{bins} \PY{o}{=} \PY{n}{np}\PY{o}{.}\PY{n}{arange}\PY{p}{(}\PY{n}{num\PYZus{}photons}\PY{o}{.}\PY{n}{min}\PY{p}{(}\PY{p}{)}\PY{p}{,} \PY{n}{num\PYZus{}photons}\PY{o}{.}\PY{n}{max}\PY{p}{(}\PY{p}{)} \PY{o}{+} \PY{l+m+mi}{2}\PY{p}{)} \PY{o}{\PYZhy{}} \PY{l+m+mf}{0.5}\PY{p}{,}
    \PY{n}{density} \PY{o}{=} \PY{k+kc}{True}\PY{p}{,}
    \PY{n}{label} \PY{o}{=} \PY{l+s+s1}{\PYZsq{}}\PY{l+s+s1}{Simulated}\PY{l+s+s1}{\PYZsq{}}\PY{p}{,}
\PY{p}{)}

\PY{k+kn}{from}\PY{+w}{ }\PY{n+nn}{scipy}\PY{+w}{ }\PY{k+kn}{import} \PY{n}{stats}
\PY{n}{lam} \PY{o}{=} \PY{n}{flux} \PY{o}{*} \PY{n}{t\PYZus{}fr}
\PY{n}{ks} \PY{o}{=} \PY{n}{np}\PY{o}{.}\PY{n}{arange}\PY{p}{(}\PY{l+m+mi}{0}\PY{p}{,} \PY{n}{num\PYZus{}photons}\PY{o}{.}\PY{n}{max}\PY{p}{(}\PY{p}{)} \PY{o}{+} \PY{l+m+mi}{1}\PY{p}{)}
\PY{n}{pmf} \PY{o}{=} \PY{n}{stats}\PY{o}{.}\PY{n}{poisson}\PY{o}{.}\PY{n}{pmf}\PY{p}{(}\PY{n}{ks}\PY{p}{,} \PY{n}{lam}\PY{p}{)}
\PY{n}{plt}\PY{o}{.}\PY{n}{plot}\PY{p}{(}
    \PY{n}{ks}\PY{p}{,}
    \PY{n}{pmf}\PY{p}{,}
    \PY{l+s+s2}{\PYZdq{}}\PY{l+s+s2}{r\PYZhy{}}\PY{l+s+s2}{\PYZdq{}}\PY{p}{,}
    \PY{n}{label} \PY{o}{=} \PY{l+s+s1}{\PYZsq{}}\PY{l+s+s1}{Poisson theory}\PY{l+s+s1}{\PYZsq{}}
\PY{p}{)}
\PY{n}{plt}\PY{o}{.}\PY{n}{yscale}\PY{p}{(}\PY{l+s+s1}{\PYZsq{}}\PY{l+s+s1}{log}\PY{l+s+s1}{\PYZsq{}}\PY{p}{)}
\PY{n}{plt}\PY{o}{.}\PY{n}{xlabel}\PY{p}{(}\PY{l+s+s1}{\PYZsq{}}\PY{l+s+s1}{Number of photons per frame}\PY{l+s+s1}{\PYZsq{}}\PY{p}{)}
\PY{n}{plt}\PY{o}{.}\PY{n}{ylabel}\PY{p}{(}\PY{l+s+s1}{\PYZsq{}}\PY{l+s+s1}{Probability / Probability density}\PY{l+s+s1}{\PYZsq{}}\PY{p}{)}
\PY{n}{plt}\PY{o}{.}\PY{n}{legend}\PY{p}{(}\PY{p}{)}
\PY{n}{plt}\PY{o}{.}\PY{n}{show}\PY{p}{(}\PY{p}{)}
\end{Verbatim}
\end{tcolorbox}

    \begin{center}
    \adjustimage{max size={0.9\linewidth}{0.9\paperheight}}{Exercise_week_03_files/Exercise_week_03_11_0.png}
    \end{center}
    { \hspace*{\fill} \\}
    
    \begin{tcolorbox}[breakable, size=fbox, boxrule=1pt, pad at break*=1mm,colback=cellbackground, colframe=cellborder]
\prompt{In}{incolor}{9}{\boxspacing}
\begin{Verbatim}[commandchars=\\\{\}]
\PY{n}{nframes} \PY{o}{=} \PY{p}{[}\PY{l+m+mi}{1}\PY{p}{,} \PY{l+m+mi}{4}\PY{p}{,} \PY{l+m+mi}{16}\PY{p}{,} \PY{l+m+mi}{64}\PY{p}{,} \PY{l+m+mi}{256}\PY{p}{,} \PY{l+m+mi}{1024}\PY{p}{]} \PY{c+c1}{\PYZsh{} number of frames for the total exposure}
\PY{n}{fig} \PY{o}{=} \PY{n}{plt}\PY{o}{.}\PY{n}{figure}\PY{p}{(}\PY{n}{figsize} \PY{o}{=} \PY{p}{(}\PY{l+m+mi}{12}\PY{p}{,} \PY{l+m+mi}{8}\PY{p}{)}\PY{p}{)}
\PY{n}{axes} \PY{o}{=} \PY{n}{fig}\PY{o}{.}\PY{n}{subplots}\PY{p}{(}\PY{l+m+mi}{2}\PY{p}{,} \PY{l+m+mi}{3}\PY{p}{)}\PY{o}{.}\PY{n}{flatten}\PY{p}{(}\PY{p}{)}

\PY{k}{for} \PY{n}{ax}\PY{p}{,} \PY{n}{n\PYZus{}fr} \PY{o+ow}{in} \PY{n+nb}{zip}\PY{p}{(}\PY{n}{axes}\PY{p}{,} \PY{n}{nframes}\PY{p}{)}\PY{p}{:}
    \PY{c+c1}{\PYZsh{} Compute the average photon count over n\PYZus{}fr frames}
    \PY{n}{counts\PYZus{}mean} \PY{o}{=} \PY{n}{np}\PY{o}{.}\PY{n}{mean}\PY{p}{(}\PY{n}{num\PYZus{}photons}\PY{o}{.}\PY{n}{reshape}\PY{p}{(}\PY{o}{\PYZhy{}}\PY{l+m+mi}{1}\PY{p}{,} \PY{n}{n\PYZus{}fr}\PY{p}{)}\PY{p}{,} \PY{n}{axis} \PY{o}{=} \PY{l+m+mi}{1}\PY{p}{)}
    \PY{n}{flux\PYZus{}measure} \PY{o}{=} \PY{n}{counts\PYZus{}mean} \PY{o}{/} \PY{n}{t\PYZus{}fr}
    \PY{n}{\PYZus{}}\PY{p}{,} \PY{n}{bins}\PY{p}{,} \PY{n}{\PYZus{}} \PY{o}{=} \PY{n}{ax}\PY{o}{.}\PY{n}{hist}\PY{p}{(}
        \PY{n}{flux\PYZus{}measure}\PY{p}{,} 
        \PY{n}{histtype} \PY{o}{=} \PY{l+s+s1}{\PYZsq{}}\PY{l+s+s1}{step}\PY{l+s+s1}{\PYZsq{}}\PY{p}{,} 
        \PY{n}{density} \PY{o}{=} \PY{k+kc}{True}\PY{p}{,} 
        \PY{n}{label} \PY{o}{=} \PY{l+s+s1}{\PYZsq{}}\PY{l+s+s1}{Simulated}\PY{l+s+s1}{\PYZsq{}}
    \PY{p}{)}

    \PY{c+c1}{\PYZsh{} theoretical mean and std for the measured flux (photon/s) from CLT}
    \PY{c+c1}{\PYZsh{} counts per frame \PYZti{} Poisson(lambda = flux * t\PYZus{}fr)}
    \PY{c+c1}{\PYZsh{} mean flux = flux}
    \PY{c+c1}{\PYZsh{} std(flux\PYZus{}measure) = sqrt(lambda / n\PYZus{}fr) / t\PYZus{}fr = sqrt(flux / (t\PYZus{}fr * n\PYZus{}fr))}
    \PY{n}{mu} \PY{o}{=} \PY{n}{flux}
    \PY{n}{sigma} \PY{o}{=} \PY{n}{np}\PY{o}{.}\PY{n}{sqrt}\PY{p}{(}\PY{n}{flux} \PY{o}{/} \PY{p}{(}\PY{n}{t\PYZus{}fr} \PY{o}{*} \PY{n}{n\PYZus{}fr}\PY{p}{)}\PY{p}{)}

    \PY{c+c1}{\PYZsh{} Overlay the theoretical Gaussian distribution}
    \PY{n}{gauss\PYZus{}x} \PY{o}{=} \PY{n}{np}\PY{o}{.}\PY{n}{linspace}\PY{p}{(}\PY{n}{bins}\PY{p}{[}\PY{l+m+mi}{0}\PY{p}{]}\PY{p}{,} \PY{n}{bins}\PY{p}{[}\PY{o}{\PYZhy{}}\PY{l+m+mi}{1}\PY{p}{]}\PY{p}{,} \PY{l+m+mi}{100}\PY{p}{)}
    \PY{n}{gauss\PYZus{}y} \PY{o}{=} \PY{p}{(}\PY{l+m+mi}{1} \PY{o}{/} \PY{p}{(}\PY{n}{sigma} \PY{o}{*} \PY{n}{np}\PY{o}{.}\PY{n}{sqrt}\PY{p}{(}\PY{l+m+mi}{2} \PY{o}{*} \PY{n}{np}\PY{o}{.}\PY{n}{pi}\PY{p}{)}\PY{p}{)}\PY{p}{)} \PY{o}{*} \PY{n}{np}\PY{o}{.}\PY{n}{exp}\PY{p}{(}\PY{o}{\PYZhy{}}\PY{l+m+mf}{0.5} \PY{o}{*} \PY{p}{(}\PY{p}{(}\PY{n}{gauss\PYZus{}x} \PY{o}{\PYZhy{}} \PY{n}{mu}\PY{p}{)} \PY{o}{/} \PY{n}{sigma}\PY{p}{)} \PY{o}{*}\PY{o}{*} \PY{l+m+mi}{2}\PY{p}{)}
    \PY{n}{ax}\PY{o}{.}\PY{n}{plot}\PY{p}{(}
        \PY{n}{gauss\PYZus{}x}\PY{p}{,}
        \PY{n}{gauss\PYZus{}y}\PY{p}{,} 
        \PY{l+s+s1}{\PYZsq{}}\PY{l+s+s1}{r\PYZhy{}}\PY{l+s+s1}{\PYZsq{}}\PY{p}{,} 
        \PY{n}{label} \PY{o}{=} \PY{l+s+s1}{\PYZsq{}}\PY{l+s+s1}{Gaussian}\PY{l+s+s1}{\PYZsq{}}
    \PY{p}{)}
    \PY{n}{ax}\PY{o}{.}\PY{n}{set\PYZus{}title}\PY{p}{(}\PY{l+s+sa}{f}\PY{l+s+s1}{\PYZsq{}}\PY{l+s+s1}{Number of frames: }\PY{l+s+si}{\PYZob{}}\PY{n}{n\PYZus{}fr}\PY{l+s+si}{\PYZcb{}}\PY{l+s+s1}{\PYZsq{}}\PY{p}{)}
    \PY{n}{ax}\PY{o}{.}\PY{n}{set\PYZus{}xlabel}\PY{p}{(}\PY{l+s+s1}{\PYZsq{}}\PY{l+s+s1}{Measured flux (photon/s)}\PY{l+s+s1}{\PYZsq{}}\PY{p}{)}
    \PY{n}{ax}\PY{o}{.}\PY{n}{set\PYZus{}ylabel}\PY{p}{(}\PY{l+s+s1}{\PYZsq{}}\PY{l+s+s1}{Probability Density}\PY{l+s+s1}{\PYZsq{}}\PY{p}{)}
    \PY{n}{ax}\PY{o}{.}\PY{n}{legend}\PY{p}{(}\PY{p}{)}

\PY{n}{plt}\PY{o}{.}\PY{n}{tight\PYZus{}layout}\PY{p}{(}\PY{p}{)}
\end{Verbatim}
\end{tcolorbox}

    \begin{center}
    \adjustimage{max size={0.9\linewidth}{0.9\paperheight}}{Exercise_week_03_files/Exercise_week_03_12_0.png}
    \end{center}
    { \hspace*{\fill} \\}
    
    \paragraph{Question 2.2: Apparent
magnitude}\label{question-2.2-apparent-magnitude}

The apparent magnitude of an astronomical source is defined as

\[m = -2.5 \log_{10}(\tilde{\phi}) + m_0\]

where \(\tilde{\phi}\) is the measured flux and \(m_0\) is a constant.
We take \(m_0 = 22.5\).

\textbf{Tasks:} 1. Using the same simulation results from Question 2.1,
plot the distribution of \(m\) for each \(N_{\rm fr}\) in
\(\{ 16, 64, 256, 1024 \}\). 2. Overlay a Gaussian distribution on each
histogram, with the standard deviation of
\(\sigma_{\rm th} = 2.5 / (\ln 10 \cdot \sqrt{\phi * N_{\rm fr} * t_{\rm fr}})\).
Derive the mean of the Gaussian distribution based on the CLT.

    \begin{tcolorbox}[breakable, size=fbox, boxrule=1pt, pad at break*=1mm,colback=cellbackground, colframe=cellborder]
\prompt{In}{incolor}{10}{\boxspacing}
\begin{Verbatim}[commandchars=\\\{\}]
\PY{n}{nframes} \PY{o}{=} \PY{p}{[}\PY{l+m+mi}{16}\PY{p}{,} \PY{l+m+mi}{64}\PY{p}{,} \PY{l+m+mi}{256}\PY{p}{,} \PY{l+m+mi}{1024}\PY{p}{]} \PY{c+c1}{\PYZsh{} number of frames for the total exposure}
\PY{n}{fig} \PY{o}{=} \PY{n}{plt}\PY{o}{.}\PY{n}{figure}\PY{p}{(}\PY{n}{figsize} \PY{o}{=} \PY{p}{(}\PY{l+m+mi}{8}\PY{p}{,} \PY{l+m+mi}{8}\PY{p}{)}\PY{p}{)}
\PY{n}{axes} \PY{o}{=} \PY{n}{fig}\PY{o}{.}\PY{n}{subplots}\PY{p}{(}\PY{l+m+mi}{2}\PY{p}{,} \PY{l+m+mi}{2}\PY{p}{)}\PY{o}{.}\PY{n}{flatten}\PY{p}{(}\PY{p}{)}

\PY{k}{for} \PY{n}{ax}\PY{p}{,} \PY{n}{n\PYZus{}fr} \PY{o+ow}{in} \PY{n+nb}{zip}\PY{p}{(}\PY{n}{axes}\PY{p}{,} \PY{n}{nframes}\PY{p}{)}\PY{p}{:}
    \PY{c+c1}{\PYZsh{} Compute the average photon count over n\PYZus{}fr frames}
    \PY{n}{counts\PYZus{}mean} \PY{o}{=} \PY{n}{np}\PY{o}{.}\PY{n}{mean}\PY{p}{(}\PY{n}{num\PYZus{}photons}\PY{o}{.}\PY{n}{reshape}\PY{p}{(}\PY{o}{\PYZhy{}}\PY{l+m+mi}{1}\PY{p}{,} \PY{n}{n\PYZus{}fr}\PY{p}{)}\PY{p}{,} \PY{n}{axis} \PY{o}{=} \PY{l+m+mi}{1}\PY{p}{)}
    \PY{n}{flux\PYZus{}measure} \PY{o}{=} \PY{n}{counts\PYZus{}mean} \PY{o}{/} \PY{n}{t\PYZus{}fr}
    \PY{n}{magnitude} \PY{o}{=} \PY{o}{\PYZhy{}}\PY{l+m+mf}{2.5} \PY{o}{*} \PY{n}{np}\PY{o}{.}\PY{n}{log10}\PY{p}{(}\PY{n}{flux\PYZus{}measure}\PY{p}{)} \PY{o}{+} \PY{l+m+mf}{22.5}
    \PY{n}{\PYZus{}}\PY{p}{,} \PY{n}{bins}\PY{p}{,} \PY{n}{\PYZus{}} \PY{o}{=} \PY{n}{ax}\PY{o}{.}\PY{n}{hist}\PY{p}{(}
        \PY{n}{magnitude}\PY{p}{,} 
        \PY{n}{histtype} \PY{o}{=} \PY{l+s+s1}{\PYZsq{}}\PY{l+s+s1}{step}\PY{l+s+s1}{\PYZsq{}}\PY{p}{,} 
        \PY{n}{density} \PY{o}{=} \PY{k+kc}{True}\PY{p}{,} 
        \PY{n}{label} \PY{o}{=} \PY{l+s+s1}{\PYZsq{}}\PY{l+s+s1}{Simulated}\PY{l+s+s1}{\PYZsq{}}
    \PY{p}{)}

    \PY{c+c1}{\PYZsh{} NOTE: EDIT BELOW to compute mu for the Gaussian overlay}
    \PY{n}{mu} \PY{o}{=} \PY{o}{\PYZhy{}}\PY{l+m+mf}{2.5} \PY{o}{*} \PY{n}{np}\PY{o}{.}\PY{n}{log10}\PY{p}{(}\PY{n}{flux}\PY{p}{)} \PY{o}{+} \PY{l+m+mf}{22.5}
    \PY{n}{sigma} \PY{o}{=} \PY{l+m+mf}{2.5} \PY{o}{/} \PY{p}{(}\PY{n}{np}\PY{o}{.}\PY{n}{log}\PY{p}{(}\PY{l+m+mi}{10}\PY{p}{)} \PY{o}{*} \PY{n}{np}\PY{o}{.}\PY{n}{sqrt}\PY{p}{(}\PY{n}{flux} \PY{o}{*} \PY{n}{t\PYZus{}fr} \PY{o}{*} \PY{n}{n\PYZus{}fr}\PY{p}{)}\PY{p}{)}

    \PY{c+c1}{\PYZsh{} Overlay the theoretical Gaussian distribution}
    \PY{n}{gauss\PYZus{}x} \PY{o}{=} \PY{n}{np}\PY{o}{.}\PY{n}{linspace}\PY{p}{(}\PY{n}{bins}\PY{p}{[}\PY{l+m+mi}{0}\PY{p}{]}\PY{p}{,} \PY{n}{bins}\PY{p}{[}\PY{o}{\PYZhy{}}\PY{l+m+mi}{1}\PY{p}{]}\PY{p}{,} \PY{l+m+mi}{100}\PY{p}{)}
    \PY{n}{gauss\PYZus{}y} \PY{o}{=} \PY{p}{(}\PY{l+m+mi}{1} \PY{o}{/} \PY{p}{(}\PY{n}{sigma} \PY{o}{*} \PY{n}{np}\PY{o}{.}\PY{n}{sqrt}\PY{p}{(}\PY{l+m+mi}{2} \PY{o}{*} \PY{n}{np}\PY{o}{.}\PY{n}{pi}\PY{p}{)}\PY{p}{)}\PY{p}{)} \PY{o}{*} \PY{n}{np}\PY{o}{.}\PY{n}{exp}\PY{p}{(}\PY{o}{\PYZhy{}}\PY{l+m+mf}{0.5} \PY{o}{*} \PY{p}{(}\PY{p}{(}\PY{n}{gauss\PYZus{}x} \PY{o}{\PYZhy{}} \PY{n}{mu}\PY{p}{)} \PY{o}{/} \PY{n}{sigma}\PY{p}{)} \PY{o}{*}\PY{o}{*} \PY{l+m+mi}{2}\PY{p}{)}
    \PY{n}{ax}\PY{o}{.}\PY{n}{plot}\PY{p}{(}
        \PY{n}{gauss\PYZus{}x}\PY{p}{,} 
        \PY{n}{gauss\PYZus{}y}\PY{p}{,} 
        \PY{l+s+s1}{\PYZsq{}}\PY{l+s+s1}{r\PYZhy{}}\PY{l+s+s1}{\PYZsq{}}\PY{p}{,} 
        \PY{n}{label} \PY{o}{=} \PY{l+s+s1}{\PYZsq{}}\PY{l+s+s1}{Gaussian}\PY{l+s+s1}{\PYZsq{}}
    \PY{p}{)}
    \PY{n}{ax}\PY{o}{.}\PY{n}{set\PYZus{}title}\PY{p}{(}\PY{l+s+sa}{f}\PY{l+s+s1}{\PYZsq{}}\PY{l+s+s1}{Number of frames: }\PY{l+s+si}{\PYZob{}}\PY{n}{n\PYZus{}fr}\PY{l+s+si}{\PYZcb{}}\PY{l+s+s1}{\PYZsq{}}\PY{p}{)}
    \PY{n}{ax}\PY{o}{.}\PY{n}{set\PYZus{}xlabel}\PY{p}{(}\PY{l+s+s1}{\PYZsq{}}\PY{l+s+s1}{Measured magnitude}\PY{l+s+s1}{\PYZsq{}}\PY{p}{)}
    \PY{n}{ax}\PY{o}{.}\PY{n}{set\PYZus{}ylabel}\PY{p}{(}\PY{l+s+s1}{\PYZsq{}}\PY{l+s+s1}{Probability Density}\PY{l+s+s1}{\PYZsq{}}\PY{p}{)}
    \PY{n}{ax}\PY{o}{.}\PY{n}{legend}\PY{p}{(}\PY{p}{)}

\PY{n}{plt}\PY{o}{.}\PY{n}{tight\PYZus{}layout}\PY{p}{(}\PY{p}{)}
\end{Verbatim}
\end{tcolorbox}

    \begin{center}
    \adjustimage{max size={0.9\linewidth}{0.9\paperheight}}{Exercise_week_03_files/Exercise_week_03_14_0.png}
    \end{center}
    { \hspace*{\fill} \\}
    
    \paragraph{Question 2.3: Brightness with transient
contamination}\label{question-2.3-brightness-with-transient-contamination}

With small probability \(\epsilon = 0.01\), a frame is hit by a bright
transient (another source) with flux
\(\phi_{\rm trans} = 1000\,{\rm photon}\,{\rm s}^{-1}\).

\textbf{Tasks:} 1. Simulate the photon counts in 256,000 frames, with
each frame having a probability \(\epsilon\) of being contaminated by
the transient source. 2. For multiple-frame exposures with
\(N_{\rm fr} = 1, 4, 16, 64, 256, 1024\), compute the measured
brightness \(\tilde{\phi}\) for each exposure. Plot the distribution of
\(\tilde{\phi}\) for each \(N_{\rm fr}\). 3. Overlay a Gaussian
distribution on each histogram, with the same mean and standard
deviation as in Question 2.1. 4. Comment on what you observe.

    \begin{tcolorbox}[breakable, size=fbox, boxrule=1pt, pad at break*=1mm,colback=cellbackground, colframe=cellborder]
\prompt{In}{incolor}{11}{\boxspacing}
\begin{Verbatim}[commandchars=\\\{\}]
\PY{n}{flux\PYZus{}transient} \PY{o}{=} \PY{l+m+mi}{1000} \PY{c+c1}{\PYZsh{} photon/s}
\PY{n}{epsilon} \PY{o}{=} \PY{l+m+mf}{0.01} \PY{c+c1}{\PYZsh{} probability of transient contamination}

\PY{c+c1}{\PYZsh{} Simulate photon counts in each frame with possible transient contamination}
\PY{n}{rng} \PY{o}{=} \PY{n}{np}\PY{o}{.}\PY{n}{random}\PY{o}{.}\PY{n}{default\PYZus{}rng}\PY{p}{(}\PY{n}{seed}\PY{p}{)}
\PY{n}{num\PYZus{}photons} \PY{o}{=} \PY{n}{rng}\PY{o}{.}\PY{n}{poisson}\PY{p}{(}\PY{n}{flux} \PY{o}{*} \PY{n}{t\PYZus{}fr}\PY{p}{,} \PY{n}{size} \PY{o}{=} \PY{n}{nsamples}\PY{p}{)}
\PY{n}{is\PYZus{}contaminated} \PY{o}{=} \PY{p}{(}\PY{n}{rng}\PY{o}{.}\PY{n}{uniform}\PY{p}{(}\PY{l+m+mi}{0}\PY{p}{,} \PY{l+m+mi}{1}\PY{p}{,} \PY{n}{size} \PY{o}{=} \PY{n}{nsamples}\PY{p}{)} \PY{o}{\PYZlt{}} \PY{n}{epsilon}\PY{p}{)}\PY{o}{.}\PY{n}{astype}\PY{p}{(}\PY{n+nb}{int}\PY{p}{)}
\PY{n}{num\PYZus{}photons\PYZus{}with\PYZus{}transient} \PY{o}{=} \PY{n}{num\PYZus{}photons} \PY{o}{+} \PY{n}{is\PYZus{}contaminated} \PY{o}{*} \PY{n}{rng}\PY{o}{.}\PY{n}{poisson}\PY{p}{(}\PY{n}{flux\PYZus{}transient} \PY{o}{*} \PY{n}{t\PYZus{}fr}\PY{p}{,} \PY{n}{size} \PY{o}{=} \PY{n}{nsamples}\PY{p}{)}
\end{Verbatim}
\end{tcolorbox}

    \begin{tcolorbox}[breakable, size=fbox, boxrule=1pt, pad at break*=1mm,colback=cellbackground, colframe=cellborder]
\prompt{In}{incolor}{12}{\boxspacing}
\begin{Verbatim}[commandchars=\\\{\}]
\PY{n}{nframes} \PY{o}{=} \PY{p}{[}\PY{l+m+mi}{1}\PY{p}{,} \PY{l+m+mi}{4}\PY{p}{,} \PY{l+m+mi}{16}\PY{p}{,} \PY{l+m+mi}{64}\PY{p}{,} \PY{l+m+mi}{256}\PY{p}{,} \PY{l+m+mi}{1024}\PY{p}{]} \PY{c+c1}{\PYZsh{} number of frames for the total exposure}
\PY{n}{fig} \PY{o}{=} \PY{n}{plt}\PY{o}{.}\PY{n}{figure}\PY{p}{(}\PY{n}{figsize} \PY{o}{=} \PY{p}{(}\PY{l+m+mi}{12}\PY{p}{,} \PY{l+m+mi}{8}\PY{p}{)}\PY{p}{)}
\PY{n}{axes} \PY{o}{=} \PY{n}{fig}\PY{o}{.}\PY{n}{subplots}\PY{p}{(}\PY{l+m+mi}{2}\PY{p}{,} \PY{l+m+mi}{3}\PY{p}{)}\PY{o}{.}\PY{n}{flatten}\PY{p}{(}\PY{p}{)}

\PY{k}{for} \PY{n}{ax}\PY{p}{,} \PY{n}{n\PYZus{}fr} \PY{o+ow}{in} \PY{n+nb}{zip}\PY{p}{(}\PY{n}{axes}\PY{p}{,} \PY{n}{nframes}\PY{p}{)}\PY{p}{:}
    \PY{c+c1}{\PYZsh{} Compute the average photon count over n\PYZus{}fr frames}
    \PY{n}{counts\PYZus{}mean} \PY{o}{=} \PY{n}{np}\PY{o}{.}\PY{n}{mean}\PY{p}{(}\PY{n}{num\PYZus{}photons\PYZus{}with\PYZus{}transient}\PY{o}{.}\PY{n}{reshape}\PY{p}{(}\PY{o}{\PYZhy{}}\PY{l+m+mi}{1}\PY{p}{,} \PY{n}{n\PYZus{}fr}\PY{p}{)}\PY{p}{,} \PY{n}{axis} \PY{o}{=} \PY{l+m+mi}{1}\PY{p}{)}
    \PY{n}{flux\PYZus{}measure} \PY{o}{=} \PY{n}{counts\PYZus{}mean} \PY{o}{/} \PY{n}{t\PYZus{}fr}
    \PY{n}{\PYZus{}}\PY{p}{,} \PY{n}{bins}\PY{p}{,} \PY{n}{\PYZus{}} \PY{o}{=} \PY{n}{ax}\PY{o}{.}\PY{n}{hist}\PY{p}{(}
        \PY{n}{flux\PYZus{}measure}\PY{p}{,} 
        \PY{n}{histtype} \PY{o}{=} \PY{l+s+s1}{\PYZsq{}}\PY{l+s+s1}{step}\PY{l+s+s1}{\PYZsq{}}\PY{p}{,} 
        \PY{n}{density} \PY{o}{=} \PY{k+kc}{True}\PY{p}{,} 
        \PY{n}{label} \PY{o}{=} \PY{l+s+s1}{\PYZsq{}}\PY{l+s+s1}{Simulated}\PY{l+s+s1}{\PYZsq{}}
    \PY{p}{)}

    \PY{c+c1}{\PYZsh{} NOTE: EDIT BELOW to compute mu and sigma for the Gaussian overlay}
    \PY{c+c1}{\PYZsh{} Theoretical mean and std for the measured flux with occasional transient contamination}
    \PY{c+c1}{\PYZsh{} Per\PYZhy{}frame expectations:}
    \PY{c+c1}{\PYZsh{}   lambda\PYZus{}bg = flux * t\PYZus{}fr}
    \PY{c+c1}{\PYZsh{}   lambda\PYZus{}trans = flux\PYZus{}transient * t\PYZus{}fr}
    \PY{c+c1}{\PYZsh{} E[counts\PYZus{}per\PYZus{}frame] = lambda\PYZus{}bg + epsilon * lambda\PYZus{}trans}
    \PY{c+c1}{\PYZsh{} Var(counts\PYZus{}per\PYZus{}frame) = lambda\PYZus{}bg + epsilon*lambda\PYZus{}trans + (epsilon \PYZhy{} epsilon**2) * lambda\PYZus{}trans**2}
    \PY{c+c1}{\PYZsh{} For averaging over n\PYZus{}fr frames, variance reduces by n\PYZus{}fr. Convert to flux (divide counts by t\PYZus{}fr).}
    \PY{n}{lambda\PYZus{}bg} \PY{o}{=} \PY{n}{flux} \PY{o}{*} \PY{n}{t\PYZus{}fr}
    \PY{n}{lambda\PYZus{}trans} \PY{o}{=} \PY{n}{flux\PYZus{}transient} \PY{o}{*} \PY{n}{t\PYZus{}fr}
    \PY{n}{mu} \PY{o}{=} \PY{n}{flux} \PY{o}{+} \PY{n}{epsilon} \PY{o}{*} \PY{n}{flux\PYZus{}transient} \PY{c+c1}{\PYZsh{} mean measured flux (photon/s)}
    \PY{n}{var\PYZus{}per\PYZus{}frame} \PY{o}{=} \PY{n}{lambda\PYZus{}bg} \PY{o}{+} \PY{n}{epsilon} \PY{o}{*} \PY{n}{lambda\PYZus{}trans} \PY{o}{+} \PY{p}{(}\PY{n}{epsilon} \PY{o}{\PYZhy{}} \PY{n}{epsilon}\PY{o}{*}\PY{o}{*}\PY{l+m+mi}{2}\PY{p}{)} \PY{o}{*} \PY{p}{(}\PY{n}{lambda\PYZus{}trans} \PY{o}{*}\PY{o}{*} \PY{l+m+mi}{2}\PY{p}{)}
    \PY{n}{sigma} \PY{o}{=} \PY{n}{np}\PY{o}{.}\PY{n}{sqrt}\PY{p}{(}\PY{n}{var\PYZus{}per\PYZus{}frame} \PY{o}{/} \PY{n}{n\PYZus{}fr}\PY{p}{)} \PY{o}{/} \PY{n}{t\PYZus{}fr} \PY{c+c1}{\PYZsh{} std dev of measured flux (photon/s)}

    \PY{c+c1}{\PYZsh{} Overlay the theoretical Gaussian distribution}
    \PY{n}{gauss\PYZus{}x} \PY{o}{=} \PY{n}{np}\PY{o}{.}\PY{n}{linspace}\PY{p}{(}\PY{n}{bins}\PY{p}{[}\PY{l+m+mi}{0}\PY{p}{]}\PY{p}{,} \PY{n}{bins}\PY{p}{[}\PY{o}{\PYZhy{}}\PY{l+m+mi}{1}\PY{p}{]}\PY{p}{,} \PY{l+m+mi}{100}\PY{p}{)}
    \PY{n}{gauss\PYZus{}y} \PY{o}{=} \PY{p}{(}\PY{l+m+mi}{1} \PY{o}{/} \PY{p}{(}\PY{n}{sigma} \PY{o}{*} \PY{n}{np}\PY{o}{.}\PY{n}{sqrt}\PY{p}{(}\PY{l+m+mi}{2} \PY{o}{*} \PY{n}{np}\PY{o}{.}\PY{n}{pi}\PY{p}{)}\PY{p}{)}\PY{p}{)} \PY{o}{*} \PY{n}{np}\PY{o}{.}\PY{n}{exp}\PY{p}{(}\PY{o}{\PYZhy{}}\PY{l+m+mf}{0.5} \PY{o}{*} \PY{p}{(}\PY{p}{(}\PY{n}{gauss\PYZus{}x} \PY{o}{\PYZhy{}} \PY{n}{mu}\PY{p}{)} \PY{o}{/} \PY{n}{sigma}\PY{p}{)} \PY{o}{*}\PY{o}{*} \PY{l+m+mi}{2}\PY{p}{)}
    \PY{n}{ax}\PY{o}{.}\PY{n}{plot}\PY{p}{(}
        \PY{n}{gauss\PYZus{}x}\PY{p}{,}
        \PY{n}{gauss\PYZus{}y}\PY{p}{,} 
        \PY{l+s+s1}{\PYZsq{}}\PY{l+s+s1}{r\PYZhy{}}\PY{l+s+s1}{\PYZsq{}}\PY{p}{,} 
        \PY{n}{label} \PY{o}{=} \PY{l+s+s1}{\PYZsq{}}\PY{l+s+s1}{Gaussian}\PY{l+s+s1}{\PYZsq{}}
    \PY{p}{)}
    \PY{n}{ax}\PY{o}{.}\PY{n}{set\PYZus{}title}\PY{p}{(}\PY{l+s+sa}{f}\PY{l+s+s1}{\PYZsq{}}\PY{l+s+s1}{Number of frames: }\PY{l+s+si}{\PYZob{}}\PY{n}{n\PYZus{}fr}\PY{l+s+si}{\PYZcb{}}\PY{l+s+s1}{\PYZsq{}}\PY{p}{)}
    \PY{n}{ax}\PY{o}{.}\PY{n}{set\PYZus{}xlabel}\PY{p}{(}\PY{l+s+s1}{\PYZsq{}}\PY{l+s+s1}{Measured flux (photon/s)}\PY{l+s+s1}{\PYZsq{}}\PY{p}{)}
    \PY{n}{ax}\PY{o}{.}\PY{n}{set\PYZus{}ylabel}\PY{p}{(}\PY{l+s+s1}{\PYZsq{}}\PY{l+s+s1}{Probability Density}\PY{l+s+s1}{\PYZsq{}}\PY{p}{)}
    \PY{n}{ax}\PY{o}{.}\PY{n}{legend}\PY{p}{(}\PY{p}{)}

\PY{n}{plt}\PY{o}{.}\PY{n}{tight\PYZus{}layout}\PY{p}{(}\PY{p}{)}
\end{Verbatim}
\end{tcolorbox}

    \begin{center}
    \adjustimage{max size={0.9\linewidth}{0.9\paperheight}}{Exercise_week_03_files/Exercise_week_03_17_0.png}
    \end{center}
    { \hspace*{\fill} \\}
    
    \subparagraph{Discussions}\label{discussions}

在这里我们明显看到,不管是 Poisson 分布本身、Poisson
分布的参数变换、Poisson 分布的叠加,都满足在极限条件下趋向于 Gauss
分布的性质. 同时,我们看到 Poisson 分布的叠加仍然是 Poisson 分布.

    \paragraph{Question 2.4: Brightness with cosmic-ray contamination
(advanced)}\label{question-2.4-brightness-with-cosmic-ray-contamination-advanced}

With small probability \(\epsilon = 0.01\), a frame is hit by a cosmic
ray, which adds a random number of photons drawn from a power-law
(Pareto) distribution with index \(\alpha = 1.5\) and minimum value
\(x_m = 10\) (do not take the distribution seriously, I made it up for
the exercise):

\[p(x) = \frac{\alpha x_m^\alpha}{x^{\alpha + 1}}, \quad x \geq x_m\]

\textbf{Tasks:} 1. Simulate the photon counts in 256,000 frames using 2
different random seeds, with each frame having a probability
\(\epsilon\) of being contaminated by a cosmic ray. 2. For
multiple-frame exposures with \(N_{\rm fr} = 1, 4, 16, 64, 256, 1024\),
compute the measured brightness \(\tilde{\phi}\) for each exposure. Plot
the distribution of \(\tilde{\phi}\) for each \(N_{\rm fr}\) for both
random seeds. 3. Comment on what you observe.

\textbf{Hint:} The Pareto distribution with parameters \(\alpha\) and
\(x_m\) can be sampled using:
\texttt{rng.pareto(a=alpha,\ size=1)\ *\ x\_m\ +\ x\_m}.

    \begin{tcolorbox}[breakable, size=fbox, boxrule=1pt, pad at break*=1mm,colback=cellbackground, colframe=cellborder]
\prompt{In}{incolor}{13}{\boxspacing}
\begin{Verbatim}[commandchars=\\\{\}]
\PY{n}{epsilon} \PY{o}{=} \PY{l+m+mf}{0.01} \PY{c+c1}{\PYZsh{} probability of cosmic\PYZhy{}ray contamination}
\PY{n}{pareto\PYZus{}alpha} \PY{o}{=} \PY{l+m+mf}{1.5} \PY{c+c1}{\PYZsh{} shape parameter for Pareto distribution}
\PY{n}{pareto\PYZus{}xm} \PY{o}{=} \PY{l+m+mi}{10} \PY{c+c1}{\PYZsh{} minimum value for Pareto distribution}

\PY{c+c1}{\PYZsh{} Simulate photon counts in each frame with possible cosmic\PYZhy{}ray contamination}
\PY{n}{rng} \PY{o}{=} \PY{n}{np}\PY{o}{.}\PY{n}{random}\PY{o}{.}\PY{n}{default\PYZus{}rng}\PY{p}{(}\PY{n}{seed}\PY{p}{)}
\PY{n}{num\PYZus{}photons} \PY{o}{=} \PY{n}{rng}\PY{o}{.}\PY{n}{poisson}\PY{p}{(}\PY{n}{flux} \PY{o}{*} \PY{n}{t\PYZus{}fr}\PY{p}{,} \PY{n}{size} \PY{o}{=} \PY{n}{nsamples}\PY{p}{)}
\PY{n}{is\PYZus{}contaminated} \PY{o}{=} \PY{p}{(}\PY{n}{rng}\PY{o}{.}\PY{n}{uniform}\PY{p}{(}\PY{l+m+mi}{0}\PY{p}{,} \PY{l+m+mi}{1}\PY{p}{,} \PY{n}{size} \PY{o}{=} \PY{n}{nsamples}\PY{p}{)} \PY{o}{\PYZlt{}} \PY{n}{epsilon}\PY{p}{)}\PY{o}{.}\PY{n}{astype}\PY{p}{(}\PY{n+nb}{int}\PY{p}{)}
\PY{c+c1}{\PYZsh{} NOTE: EDIT HERE to add Pareto\PYZhy{}distributed contamination}
\PY{n}{num\PYZus{}photons\PYZus{}with\PYZus{}cr\PYZus{}1} \PY{o}{=} \PY{n}{num\PYZus{}photons} \PY{o}{+} \PY{n}{is\PYZus{}contaminated} \PY{o}{*} \PY{p}{(}\PY{n}{rng}\PY{o}{.}\PY{n}{pareto}\PY{p}{(}\PY{n}{pareto\PYZus{}alpha}\PY{p}{,} \PY{n}{size} \PY{o}{=} \PY{n}{nsamples}\PY{p}{)} \PY{o}{*} \PY{n}{pareto\PYZus{}xm} \PY{o}{+} \PY{n}{pareto\PYZus{}xm}\PY{p}{)}

\PY{c+c1}{\PYZsh{} Run another simulation with random seed 42}
\PY{n}{rng} \PY{o}{=} \PY{n}{np}\PY{o}{.}\PY{n}{random}\PY{o}{.}\PY{n}{default\PYZus{}rng}\PY{p}{(}\PY{l+m+mi}{42}\PY{p}{)}
\PY{n}{num\PYZus{}photons} \PY{o}{=} \PY{n}{rng}\PY{o}{.}\PY{n}{poisson}\PY{p}{(}\PY{n}{flux} \PY{o}{*} \PY{n}{t\PYZus{}fr}\PY{p}{,} \PY{n}{size} \PY{o}{=} \PY{n}{nsamples}\PY{p}{)}
\PY{n}{is\PYZus{}contaminated} \PY{o}{=} \PY{p}{(}\PY{n}{rng}\PY{o}{.}\PY{n}{uniform}\PY{p}{(}\PY{l+m+mi}{0}\PY{p}{,} \PY{l+m+mi}{1}\PY{p}{,} \PY{n}{size} \PY{o}{=} \PY{n}{nsamples}\PY{p}{)} \PY{o}{\PYZlt{}} \PY{n}{epsilon}\PY{p}{)}\PY{o}{.}\PY{n}{astype}\PY{p}{(}\PY{n+nb}{int}\PY{p}{)}
\PY{c+c1}{\PYZsh{} NOTE: EDIT HERE to add Pareto\PYZhy{}distributed contamination}
\PY{n}{num\PYZus{}photons\PYZus{}with\PYZus{}cr\PYZus{}2} \PY{o}{=} \PY{n}{num\PYZus{}photons} \PY{o}{+} \PY{n}{is\PYZus{}contaminated} \PY{o}{*} \PY{p}{(}\PY{n}{rng}\PY{o}{.}\PY{n}{pareto}\PY{p}{(}\PY{n}{pareto\PYZus{}alpha}\PY{p}{,} \PY{n}{size} \PY{o}{=} \PY{n}{nsamples}\PY{p}{)} \PY{o}{*} \PY{n}{pareto\PYZus{}xm} \PY{o}{+} \PY{n}{pareto\PYZus{}xm}\PY{p}{)}
\end{Verbatim}
\end{tcolorbox}

    \begin{tcolorbox}[breakable, size=fbox, boxrule=1pt, pad at break*=1mm,colback=cellbackground, colframe=cellborder]
\prompt{In}{incolor}{14}{\boxspacing}
\begin{Verbatim}[commandchars=\\\{\}]
\PY{n}{nframes} \PY{o}{=} \PY{p}{[}\PY{l+m+mi}{1}\PY{p}{,} \PY{l+m+mi}{4}\PY{p}{,} \PY{l+m+mi}{16}\PY{p}{,} \PY{l+m+mi}{64}\PY{p}{,} \PY{l+m+mi}{256}\PY{p}{,} \PY{l+m+mi}{1024}\PY{p}{]} \PY{c+c1}{\PYZsh{} number of frames for the total exposure}
\PY{n}{fig} \PY{o}{=} \PY{n}{plt}\PY{o}{.}\PY{n}{figure}\PY{p}{(}\PY{n}{figsize} \PY{o}{=} \PY{p}{(}\PY{l+m+mi}{12}\PY{p}{,} \PY{l+m+mi}{8}\PY{p}{)}\PY{p}{)}
\PY{n}{axes} \PY{o}{=} \PY{n}{fig}\PY{o}{.}\PY{n}{subplots}\PY{p}{(}\PY{l+m+mi}{2}\PY{p}{,} \PY{l+m+mi}{3}\PY{p}{)}\PY{o}{.}\PY{n}{flatten}\PY{p}{(}\PY{p}{)}

\PY{k}{for} \PY{n}{ax}\PY{p}{,} \PY{n}{n\PYZus{}fr} \PY{o+ow}{in} \PY{n+nb}{zip}\PY{p}{(}\PY{n}{axes}\PY{p}{,} \PY{n}{nframes}\PY{p}{)}\PY{p}{:}
    \PY{c+c1}{\PYZsh{} Compute the average photon count over n\PYZus{}fr frames}
    \PY{n}{counts\PYZus{}mean} \PY{o}{=} \PY{n}{np}\PY{o}{.}\PY{n}{mean}\PY{p}{(}\PY{n}{num\PYZus{}photons\PYZus{}with\PYZus{}cr\PYZus{}1}\PY{o}{.}\PY{n}{reshape}\PY{p}{(}\PY{o}{\PYZhy{}}\PY{l+m+mi}{1}\PY{p}{,} \PY{n}{n\PYZus{}fr}\PY{p}{)}\PY{p}{,} \PY{n}{axis} \PY{o}{=} \PY{l+m+mi}{1}\PY{p}{)}
    \PY{n}{flux\PYZus{}measure} \PY{o}{=} \PY{n}{counts\PYZus{}mean} \PY{o}{/} \PY{n}{t\PYZus{}fr}
    \PY{n}{ax}\PY{o}{.}\PY{n}{hist}\PY{p}{(}
        \PY{n}{flux\PYZus{}measure}\PY{p}{,} 
        \PY{n}{histtype} \PY{o}{=} \PY{l+s+s1}{\PYZsq{}}\PY{l+s+s1}{step}\PY{l+s+s1}{\PYZsq{}}\PY{p}{,} 
        \PY{n}{density} \PY{o}{=} \PY{k+kc}{True}\PY{p}{,} 
        \PY{n}{label} \PY{o}{=} \PY{l+s+sa}{f}\PY{l+s+s1}{\PYZsq{}}\PY{l+s+s1}{seed = }\PY{l+s+si}{\PYZob{}}\PY{n}{seed}\PY{l+s+si}{\PYZcb{}}\PY{l+s+s1}{\PYZsq{}}
    \PY{p}{)}

    \PY{n}{counts\PYZus{}mean} \PY{o}{=} \PY{n}{np}\PY{o}{.}\PY{n}{mean}\PY{p}{(}\PY{n}{num\PYZus{}photons\PYZus{}with\PYZus{}cr\PYZus{}2}\PY{o}{.}\PY{n}{reshape}\PY{p}{(}\PY{o}{\PYZhy{}}\PY{l+m+mi}{1}\PY{p}{,} \PY{n}{n\PYZus{}fr}\PY{p}{)}\PY{p}{,} \PY{n}{axis} \PY{o}{=} \PY{l+m+mi}{1}\PY{p}{)}
    \PY{n}{flux\PYZus{}measure} \PY{o}{=} \PY{n}{counts\PYZus{}mean} \PY{o}{/} \PY{n}{t\PYZus{}fr}
    \PY{n}{ax}\PY{o}{.}\PY{n}{hist}\PY{p}{(}
        \PY{n}{flux\PYZus{}measure}\PY{p}{,} 
        \PY{n}{histtype} \PY{o}{=} \PY{l+s+s1}{\PYZsq{}}\PY{l+s+s1}{step}\PY{l+s+s1}{\PYZsq{}}\PY{p}{,} 
        \PY{n}{density} \PY{o}{=} \PY{k+kc}{True}\PY{p}{,} 
        \PY{n}{label} \PY{o}{=} \PY{l+s+s1}{\PYZsq{}}\PY{l+s+s1}{seed = 42}\PY{l+s+s1}{\PYZsq{}}
    \PY{p}{)}

    \PY{n}{ax}\PY{o}{.}\PY{n}{set\PYZus{}yscale}\PY{p}{(}\PY{l+s+s1}{\PYZsq{}}\PY{l+s+s1}{log}\PY{l+s+s1}{\PYZsq{}}\PY{p}{)}
    \PY{n}{ax}\PY{o}{.}\PY{n}{set\PYZus{}title}\PY{p}{(}\PY{l+s+sa}{f}\PY{l+s+s1}{\PYZsq{}}\PY{l+s+s1}{Number of frames: }\PY{l+s+si}{\PYZob{}}\PY{n}{n\PYZus{}fr}\PY{l+s+si}{\PYZcb{}}\PY{l+s+s1}{\PYZsq{}}\PY{p}{)}
    \PY{n}{ax}\PY{o}{.}\PY{n}{set\PYZus{}xlabel}\PY{p}{(}\PY{l+s+s1}{\PYZsq{}}\PY{l+s+s1}{Measured flux (photon/s)}\PY{l+s+s1}{\PYZsq{}}\PY{p}{)}
    \PY{n}{ax}\PY{o}{.}\PY{n}{set\PYZus{}ylabel}\PY{p}{(}\PY{l+s+s1}{\PYZsq{}}\PY{l+s+s1}{Probability Density}\PY{l+s+s1}{\PYZsq{}}\PY{p}{)}
    \PY{n}{ax}\PY{o}{.}\PY{n}{legend}\PY{p}{(}\PY{p}{)}

\PY{n}{plt}\PY{o}{.}\PY{n}{tight\PYZus{}layout}\PY{p}{(}\PY{p}{)}
\end{Verbatim}
\end{tcolorbox}

    \begin{center}
    \adjustimage{max size={0.9\linewidth}{0.9\paperheight}}{Exercise_week_03_files/Exercise_week_03_21_0.png}
    \end{center}
    { \hspace*{\fill} \\}
    
    \subparagraph{Discussions}\label{discussions}

不同种子生成的两组数据图像类似,而且都在数据量更大的情况下体现出更平缓的特征.

    \subsubsection{Exercise 3: Toy universe}\label{exercise-3-toy-universe}

Build a spherical \emph{toy universe} of radius \(R=1\) in Euclidean 3D
space. Assume a coherent unit system (you may omit units). The universe
is uniformly filled with stars (constant number density), and the
observer is at the center (the origin).

\paragraph{Question 3.1: Radial distribution of
stars}\label{question-3.1-radial-distribution-of-stars}

Generate \(N=10,000\) stars uniformly distributed in the universe. There
are two sampling methods: 1. Draw Cartesian coordinates \((x,y,z)\)
uniformly in the cube \((-1,1)^3\) and reject points outside the unit
sphere. 2. Sample the spherical coordinates \((r, \theta, \phi)\)
directly.

\textbf{Tasks:} 1. Generate Cartesian coordinates of \(N\) stars. How
many uniform random numbers do you actually consume and how is your
acceptance rate compared to the theoretical expectation? 2. Plot the
histogram of the distances of the stars from the origin (\(r\)) and
overlay the theoretical distribution of distances. 3. Since the universe
is isotropic and we only care about the \textbf{radial} distribution,
sample just \(r\) using the variable transformation method. Validate
that the sampled \(r\) matches the theoretical distribution. 4. For both
datasets (from tasks 1 and 3), validate uniformity of stellar
distribution by plotting the radial number density in concentric
spherical shells with equal thickness, and overlaying the theoretical
expectation.

    \begin{tcolorbox}[breakable, size=fbox, boxrule=1pt, pad at break*=1mm,colback=cellbackground, colframe=cellborder]
\prompt{In}{incolor}{15}{\boxspacing}
\begin{Verbatim}[commandchars=\\\{\}]
\PY{c+c1}{\PYZsh{} Set the random seed}
\PY{n}{seed} \PY{o}{=} \PY{l+m+mi}{2024011182} \PY{c+c1}{\PYZsh{} NOTE: EDIT HERE to insert your student ID as the seed}

\PY{n}{nstars} \PY{o}{=} \PY{l+m+mi}{10000} \PY{c+c1}{\PYZsh{} Number of stars to sample}
\PY{n}{nsamples} \PY{o}{=} \PY{l+m+mi}{10} \PY{o}{*} \PY{n}{nstars} \PY{c+c1}{\PYZsh{} Number of samples to draw at once}

\PY{n}{rng} \PY{o}{=} \PY{n}{np}\PY{o}{.}\PY{n}{random}\PY{o}{.}\PY{n}{default\PYZus{}rng}\PY{p}{(}\PY{n}{seed}\PY{p}{)}

\PY{c+c1}{\PYZsh{} Sample uniform Cartesian coordinates}
\PY{n}{x} \PY{o}{=} \PY{n}{rng}\PY{o}{.}\PY{n}{uniform}\PY{p}{(}\PY{o}{\PYZhy{}}\PY{l+m+mi}{1}\PY{p}{,} \PY{l+m+mi}{1}\PY{p}{,} \PY{n}{size} \PY{o}{=} \PY{n}{nsamples}\PY{p}{)}
\PY{n}{y} \PY{o}{=} \PY{n}{rng}\PY{o}{.}\PY{n}{uniform}\PY{p}{(}\PY{o}{\PYZhy{}}\PY{l+m+mi}{1}\PY{p}{,} \PY{l+m+mi}{1}\PY{p}{,} \PY{n}{size} \PY{o}{=} \PY{n}{nsamples}\PY{p}{)}
\PY{n}{z} \PY{o}{=} \PY{n}{rng}\PY{o}{.}\PY{n}{uniform}\PY{p}{(}\PY{o}{\PYZhy{}}\PY{l+m+mi}{1}\PY{p}{,} \PY{l+m+mi}{1}\PY{p}{,} \PY{n}{size} \PY{o}{=} \PY{n}{nsamples}\PY{p}{)}

\PY{c+c1}{\PYZsh{} Reject points outside the unit sphere}
\PY{n}{r} \PY{o}{=} \PY{n}{np}\PY{o}{.}\PY{n}{sqrt}\PY{p}{(}\PY{n}{x}\PY{o}{*}\PY{o}{*}\PY{l+m+mi}{2} \PY{o}{+} \PY{n}{y}\PY{o}{*}\PY{o}{*}\PY{l+m+mi}{2} \PY{o}{+} \PY{n}{z}\PY{o}{*}\PY{o}{*}\PY{l+m+mi}{2}\PY{p}{)}
\PY{n}{mask} \PY{o}{=} \PY{p}{(}\PY{n}{r} \PY{o}{\PYZlt{}} \PY{l+m+mi}{1}\PY{p}{)}
\PY{n}{r} \PY{o}{=} \PY{n}{r}\PY{p}{[}\PY{n}{mask}\PY{p}{]}\PY{p}{[}\PY{p}{:}\PY{n}{nstars}\PY{p}{]} \PY{c+c1}{\PYZsh{} Keep only the first N stars}

\PY{c+c1}{\PYZsh{} Compute the acceptance ratio upon 10000 stars being accepted}
\PY{n}{accept\PYZus{}index} \PY{o}{=} \PY{n}{np}\PY{o}{.}\PY{n}{flatnonzero}\PY{p}{(}\PY{n}{mask}\PY{p}{)}
\PY{n}{used\PYZus{}index} \PY{o}{=} \PY{n}{accept\PYZus{}index}\PY{p}{[}\PY{n}{nstars} \PY{o}{\PYZhy{}} \PY{l+m+mi}{1}\PY{p}{]} \PY{o}{+} \PY{l+m+mi}{1}
\PY{n}{random\PYZus{}draws} \PY{o}{=} \PY{n}{used\PYZus{}index} \PY{o}{*} \PY{l+m+mi}{3}  \PY{c+c1}{\PYZsh{} 3 coordinates per star}
\PY{n}{acceptance\PYZus{}ratio} \PY{o}{=} \PY{n}{nstars} \PY{o}{/} \PY{n}{random\PYZus{}draws}
\PY{c+c1}{\PYZsh{} NOTE: EDIT HERE to compute the theoretical acceptance ratio}
\PY{n}{acceptance\PYZus{}ratio\PYZus{}theory} \PY{o}{=} \PY{p}{(}\PY{l+m+mi}{4} \PY{o}{*} \PY{n}{np}\PY{o}{.}\PY{n}{pi} \PY{o}{/} \PY{l+m+mi}{3}\PY{p}{)} \PY{o}{/} \PY{p}{(}\PY{l+m+mi}{2}\PY{o}{*}\PY{o}{*}\PY{l+m+mi}{3}\PY{p}{)}

\PY{n+nb}{print}\PY{p}{(}\PY{l+s+sa}{f}\PY{l+s+s2}{\PYZdq{}}\PY{l+s+s2}{Acceptance ratio: }\PY{l+s+si}{\PYZob{}}\PY{n}{acceptance\PYZus{}ratio}\PY{l+s+si}{:}\PY{l+s+s2}{.4\PYZpc{}}\PY{l+s+si}{\PYZcb{}}\PY{l+s+s2}{\PYZdq{}}\PY{p}{)}
\PY{n+nb}{print}\PY{p}{(}\PY{l+s+sa}{f}\PY{l+s+s2}{\PYZdq{}}\PY{l+s+s2}{Theoretical acceptance ratio: }\PY{l+s+si}{\PYZob{}}\PY{n}{acceptance\PYZus{}ratio\PYZus{}theory}\PY{l+s+si}{:}\PY{l+s+s2}{.4\PYZpc{}}\PY{l+s+si}{\PYZcb{}}\PY{l+s+s2}{\PYZdq{}}\PY{p}{)}
\end{Verbatim}
\end{tcolorbox}

    \begin{Verbatim}[commandchars=\\\{\}]
Acceptance ratio: 17.4438\%
Theoretical acceptance ratio: 52.3599\%
    \end{Verbatim}

    \begin{tcolorbox}[breakable, size=fbox, boxrule=1pt, pad at break*=1mm,colback=cellbackground, colframe=cellborder]
\prompt{In}{incolor}{16}{\boxspacing}
\begin{Verbatim}[commandchars=\\\{\}]
\PY{c+c1}{\PYZsh{} Plot the histogram of r}
\PY{n}{plt}\PY{o}{.}\PY{n}{hist}\PY{p}{(}
    \PY{n}{r}\PY{p}{,} 
    \PY{n}{histtype} \PY{o}{=} \PY{l+s+s1}{\PYZsq{}}\PY{l+s+s1}{step}\PY{l+s+s1}{\PYZsq{}}\PY{p}{,}
    \PY{n}{density} \PY{o}{=} \PY{k+kc}{True}\PY{p}{,}
    \PY{n}{label} \PY{o}{=} \PY{l+s+s1}{\PYZsq{}}\PY{l+s+s1}{Simulated}\PY{l+s+s1}{\PYZsq{}}
\PY{p}{)}

\PY{c+c1}{\PYZsh{} Overlay the theoretical distribution}
\PY{n}{r\PYZus{}theory} \PY{o}{=} \PY{n}{np}\PY{o}{.}\PY{n}{linspace}\PY{p}{(}\PY{l+m+mi}{0}\PY{p}{,} \PY{l+m+mi}{1}\PY{p}{,} \PY{l+m+mi}{100}\PY{p}{)}
\PY{c+c1}{\PYZsh{} NOTE: EDIT HERE to compute the theoretical PDF of r}
\PY{n}{p\PYZus{}r\PYZus{}theory} \PY{o}{=} \PY{l+m+mi}{3} \PY{o}{*} \PY{n}{r\PYZus{}theory}\PY{o}{*}\PY{o}{*}\PY{l+m+mi}{2}
\PY{n}{plt}\PY{o}{.}\PY{n}{plot}\PY{p}{(}
    \PY{n}{r\PYZus{}theory}\PY{p}{,}
    \PY{n}{p\PYZus{}r\PYZus{}theory}\PY{p}{,}
    \PY{l+s+s1}{\PYZsq{}}\PY{l+s+s1}{r\PYZhy{}}\PY{l+s+s1}{\PYZsq{}}\PY{p}{,} 
    \PY{n}{label} \PY{o}{=} \PY{l+s+s1}{\PYZsq{}}\PY{l+s+s1}{Theory}\PY{l+s+s1}{\PYZsq{}}
\PY{p}{)}

\PY{n}{plt}\PY{o}{.}\PY{n}{xlabel}\PY{p}{(}\PY{l+s+s1}{\PYZsq{}}\PY{l+s+s1}{Radius \PYZdl{}r\PYZdl{}}\PY{l+s+s1}{\PYZsq{}}\PY{p}{)}
\PY{n}{plt}\PY{o}{.}\PY{n}{ylabel}\PY{p}{(}\PY{l+s+s1}{\PYZsq{}}\PY{l+s+s1}{Probability Density}\PY{l+s+s1}{\PYZsq{}}\PY{p}{)}
\PY{n}{plt}\PY{o}{.}\PY{n}{legend}\PY{p}{(}\PY{p}{)}
\PY{n}{plt}\PY{o}{.}\PY{n}{show}\PY{p}{(}\PY{p}{)}
\end{Verbatim}
\end{tcolorbox}

    \begin{center}
    \adjustimage{max size={0.9\linewidth}{0.9\paperheight}}{Exercise_week_03_files/Exercise_week_03_25_0.png}
    \end{center}
    { \hspace*{\fill} \\}
    
    \begin{tcolorbox}[breakable, size=fbox, boxrule=1pt, pad at break*=1mm,colback=cellbackground, colframe=cellborder]
\prompt{In}{incolor}{17}{\boxspacing}
\begin{Verbatim}[commandchars=\\\{\}]
\PY{c+c1}{\PYZsh{} Sample r from the theoretical distribution directly}
\PY{n}{rng} \PY{o}{=} \PY{n}{np}\PY{o}{.}\PY{n}{random}\PY{o}{.}\PY{n}{default\PYZus{}rng}\PY{p}{(}\PY{n}{seed}\PY{p}{)}
\PY{n}{u} \PY{o}{=} \PY{n}{rng}\PY{o}{.}\PY{n}{uniform}\PY{p}{(}\PY{l+m+mi}{0}\PY{p}{,} \PY{l+m+mi}{1}\PY{p}{,} \PY{n}{size} \PY{o}{=} \PY{n}{nstars}\PY{p}{)} 
\PY{c+c1}{\PYZsh{} NOTE: EDIT HERE to compute r from u}
\PY{n}{r\PYZus{}sampled} \PY{o}{=} \PY{n}{u}\PY{o}{*}\PY{o}{*}\PY{p}{(}\PY{l+m+mi}{1}\PY{o}{/}\PY{l+m+mi}{3}\PY{p}{)}

\PY{c+c1}{\PYZsh{} Validate that the sampled r matches the theoretical distribution}
\PY{n}{plt}\PY{o}{.}\PY{n}{hist}\PY{p}{(}
    \PY{n}{r\PYZus{}sampled}\PY{p}{,}
    \PY{n}{histtype} \PY{o}{=} \PY{l+s+s1}{\PYZsq{}}\PY{l+s+s1}{step}\PY{l+s+s1}{\PYZsq{}}\PY{p}{,}
    \PY{n}{density} \PY{o}{=} \PY{k+kc}{True}\PY{p}{,}
    \PY{n}{label} \PY{o}{=} \PY{l+s+s1}{\PYZsq{}}\PY{l+s+s1}{Sampled}\PY{l+s+s1}{\PYZsq{}}
\PY{p}{)}
\PY{n}{plt}\PY{o}{.}\PY{n}{plot}\PY{p}{(}
    \PY{n}{r\PYZus{}theory}\PY{p}{,}
    \PY{n}{p\PYZus{}r\PYZus{}theory}\PY{p}{,}
    \PY{l+s+s1}{\PYZsq{}}\PY{l+s+s1}{r\PYZhy{}}\PY{l+s+s1}{\PYZsq{}}\PY{p}{,}
    \PY{n}{label} \PY{o}{=} \PY{l+s+s1}{\PYZsq{}}\PY{l+s+s1}{Theory}\PY{l+s+s1}{\PYZsq{}}
\PY{p}{)}

\PY{n}{plt}\PY{o}{.}\PY{n}{xlabel}\PY{p}{(}\PY{l+s+s1}{\PYZsq{}}\PY{l+s+s1}{Radius \PYZdl{}r\PYZdl{}}\PY{l+s+s1}{\PYZsq{}}\PY{p}{)}
\PY{n}{plt}\PY{o}{.}\PY{n}{ylabel}\PY{p}{(}\PY{l+s+s1}{\PYZsq{}}\PY{l+s+s1}{Probability Density}\PY{l+s+s1}{\PYZsq{}}\PY{p}{)}
\PY{n}{plt}\PY{o}{.}\PY{n}{legend}\PY{p}{(}\PY{p}{)}
\PY{n}{plt}\PY{o}{.}\PY{n}{show}\PY{p}{(}\PY{p}{)}
\end{Verbatim}
\end{tcolorbox}

    \begin{center}
    \adjustimage{max size={0.9\linewidth}{0.9\paperheight}}{Exercise_week_03_files/Exercise_week_03_26_0.png}
    \end{center}
    { \hspace*{\fill} \\}
    
    \begin{tcolorbox}[breakable, size=fbox, boxrule=1pt, pad at break*=1mm,colback=cellbackground, colframe=cellborder]
\prompt{In}{incolor}{18}{\boxspacing}
\begin{Verbatim}[commandchars=\\\{\}]
\PY{c+c1}{\PYZsh{} Define spherical shells with equal thickness}
\PY{n}{r\PYZus{}edges} \PY{o}{=} \PY{n}{np}\PY{o}{.}\PY{n}{linspace}\PY{p}{(}\PY{l+m+mi}{0}\PY{p}{,} \PY{l+m+mi}{1}\PY{p}{,} \PY{l+m+mi}{11}\PY{p}{)} \PY{c+c1}{\PYZsh{} 10 shells}
\PY{n}{r\PYZus{}centers} \PY{o}{=} \PY{l+m+mf}{0.5} \PY{o}{*} \PY{p}{(}\PY{n}{r\PYZus{}edges}\PY{p}{[}\PY{p}{:}\PY{o}{\PYZhy{}}\PY{l+m+mi}{1}\PY{p}{]} \PY{o}{+} \PY{n}{r\PYZus{}edges}\PY{p}{[}\PY{l+m+mi}{1}\PY{p}{:}\PY{p}{]}\PY{p}{)}
\PY{n}{shell\PYZus{}volumes} \PY{o}{=} \PY{p}{(}\PY{l+m+mi}{4}\PY{o}{/}\PY{l+m+mi}{3}\PY{p}{)} \PY{o}{*} \PY{n}{np}\PY{o}{.}\PY{n}{pi} \PY{o}{*} \PY{p}{(}\PY{n}{r\PYZus{}edges}\PY{p}{[}\PY{l+m+mi}{1}\PY{p}{:}\PY{p}{]}\PY{o}{*}\PY{o}{*}\PY{l+m+mi}{3} \PY{o}{\PYZhy{}} \PY{n}{r\PYZus{}edges}\PY{p}{[}\PY{p}{:}\PY{o}{\PYZhy{}}\PY{l+m+mi}{1}\PY{p}{]}\PY{o}{*}\PY{o}{*}\PY{l+m+mi}{3}\PY{p}{)}

\PY{c+c1}{\PYZsh{} Compute counts in each shell from the sampled radii and estimate Poisson errors}
\PY{n}{shell\PYZus{}counts}\PY{p}{,} \PY{n}{\PYZus{}} \PY{o}{=} \PY{n}{np}\PY{o}{.}\PY{n}{histogram}\PY{p}{(}\PY{n}{r\PYZus{}sampled}\PY{p}{,} \PY{n}{bins} \PY{o}{=} \PY{n}{r\PYZus{}edges}\PY{p}{)}
\PY{n}{shell\PYZus{}densities} \PY{o}{=} \PY{n}{shell\PYZus{}counts} \PY{o}{/} \PY{n}{shell\PYZus{}volumes}

\PY{n}{plt}\PY{o}{.}\PY{n}{step}\PY{p}{(}\PY{n}{r\PYZus{}centers}\PY{p}{,} \PY{n}{shell\PYZus{}densities}\PY{p}{,} \PY{n}{where} \PY{o}{=} \PY{l+s+s1}{\PYZsq{}}\PY{l+s+s1}{mid}\PY{l+s+s1}{\PYZsq{}}\PY{p}{,} \PY{n}{label} \PY{o}{=} \PY{l+s+s1}{\PYZsq{}}\PY{l+s+s1}{Simulated}\PY{l+s+s1}{\PYZsq{}}\PY{p}{)}

\PY{c+c1}{\PYZsh{} Overlay the theoretical constant density}
\PY{n}{density\PYZus{}theory} \PY{o}{=} \PY{n}{nstars} \PY{o}{/} \PY{p}{(}\PY{l+m+mi}{4} \PY{o}{*} \PY{n}{np}\PY{o}{.}\PY{n}{pi} \PY{o}{/} \PY{l+m+mi}{3}\PY{p}{)}
\PY{n}{plt}\PY{o}{.}\PY{n}{axhline}\PY{p}{(}
    \PY{n}{density\PYZus{}theory}\PY{p}{,}
    \PY{n}{color} \PY{o}{=} \PY{l+s+s1}{\PYZsq{}}\PY{l+s+s1}{r}\PY{l+s+s1}{\PYZsq{}}\PY{p}{,}
    \PY{n}{linestyle} \PY{o}{=} \PY{l+s+s1}{\PYZsq{}}\PY{l+s+s1}{\PYZhy{}\PYZhy{}}\PY{l+s+s1}{\PYZsq{}}\PY{p}{,}
    \PY{n}{label} \PY{o}{=} \PY{l+s+s1}{\PYZsq{}}\PY{l+s+s1}{Theory}\PY{l+s+s1}{\PYZsq{}}
\PY{p}{)}

\PY{n}{plt}\PY{o}{.}\PY{n}{xlabel}\PY{p}{(}\PY{l+s+s1}{\PYZsq{}}\PY{l+s+s1}{Radius \PYZdl{}r\PYZdl{}}\PY{l+s+s1}{\PYZsq{}}\PY{p}{)}
\PY{n}{plt}\PY{o}{.}\PY{n}{ylabel}\PY{p}{(}\PY{l+s+s1}{\PYZsq{}}\PY{l+s+s1}{Stellar Density}\PY{l+s+s1}{\PYZsq{}}\PY{p}{)}
\PY{n}{plt}\PY{o}{.}\PY{n}{legend}\PY{p}{(}\PY{p}{)}
\PY{n}{plt}\PY{o}{.}\PY{n}{show}\PY{p}{(}\PY{p}{)}
\end{Verbatim}
\end{tcolorbox}

    \begin{center}
    \adjustimage{max size={0.9\linewidth}{0.9\paperheight}}{Exercise_week_03_files/Exercise_week_03_27_0.png}
    \end{center}
    { \hspace*{\fill} \\}
    
    \paragraph{Question 3.2: Distribution of stellar counts and
densities}\label{question-3.2-distribution-of-stellar-counts-and-densities}

\textbf{Tasks:} 1. Rerun the simulation in Question 3.1 with 1000
different random seeds, each time generating \(N=10,000\) stars using
the variable transformation method. 2. For each simulation, count stars
within radius \(r < 0.8\). Plot the distribution of counts and overlay
Binomial and Poisson distributions with appropriate parameters. 3. Count
stars within \(r < 0.1\). Plot the distribution of stellar counts over
the 1000 simulations, and overlay Binomial and Poisson distributions
with appropriate parameters. 4. Given your observations from tasks 2 and
3, replot the stellar density distribution in spherical shells from
Question 3.1 with appropriate error bars. Comment on your results.

    \begin{tcolorbox}[breakable, size=fbox, boxrule=1pt, pad at break*=1mm,colback=cellbackground, colframe=cellborder]
\prompt{In}{incolor}{19}{\boxspacing}
\begin{Verbatim}[commandchars=\\\{\}]
\PY{n}{seeds} \PY{o}{=} \PY{p}{[}\PY{n}{seed} \PY{o}{+} \PY{n}{i} \PY{k}{for} \PY{n}{i} \PY{o+ow}{in} \PY{n+nb}{range}\PY{p}{(}\PY{l+m+mi}{1000}\PY{p}{)}\PY{p}{]} \PY{c+c1}{\PYZsh{} Different seeds based on student ID}
\PY{n}{r\PYZus{}samples} \PY{o}{=} \PY{p}{[}\PY{p}{]}

\PY{c+c1}{\PYZsh{} Simulate the toy universe with different seeds}
\PY{k}{for} \PY{n}{s} \PY{o+ow}{in} \PY{n}{seeds}\PY{p}{:}
    \PY{n}{rng} \PY{o}{=} \PY{n}{np}\PY{o}{.}\PY{n}{random}\PY{o}{.}\PY{n}{default\PYZus{}rng}\PY{p}{(}\PY{n}{s}\PY{p}{)}
    \PY{n}{u} \PY{o}{=} \PY{n}{rng}\PY{o}{.}\PY{n}{uniform}\PY{p}{(}\PY{l+m+mi}{0}\PY{p}{,} \PY{l+m+mi}{1}\PY{p}{,} \PY{n}{size} \PY{o}{=} \PY{n}{nstars}\PY{p}{)}
    \PY{c+c1}{\PYZsh{} NOTE: EDIT HERE to compute r from u}
    \PY{n}{r} \PY{o}{=} \PY{n}{u}\PY{o}{*}\PY{o}{*}\PY{p}{(}\PY{l+m+mi}{1}\PY{o}{/}\PY{l+m+mi}{3}\PY{p}{)}
    \PY{n}{r\PYZus{}samples}\PY{o}{.}\PY{n}{append}\PY{p}{(}\PY{n}{r}\PY{p}{)}
\end{Verbatim}
\end{tcolorbox}

    \begin{tcolorbox}[breakable, size=fbox, boxrule=1pt, pad at break*=1mm,colback=cellbackground, colframe=cellborder]
\prompt{In}{incolor}{20}{\boxspacing}
\begin{Verbatim}[commandchars=\\\{\}]
\PY{k+kn}{from}\PY{+w}{ }\PY{n+nn}{scipy}\PY{n+nn}{.}\PY{n+nn}{stats}\PY{+w}{ }\PY{k+kn}{import} \PY{n}{binom}\PY{p}{,} \PY{n}{poisson}

\PY{c+c1}{\PYZsh{} Count stars within r \PYZlt{} 0.8}
\PY{n}{r\PYZus{}thres\PYZus{}1} \PY{o}{=} \PY{l+m+mf}{0.8}
\PY{n}{counts\PYZus{}1} \PY{o}{=} \PY{p}{[}\PY{n}{np}\PY{o}{.}\PY{n}{sum}\PY{p}{(}\PY{n}{r} \PY{o}{\PYZlt{}} \PY{n}{r\PYZus{}thres\PYZus{}1}\PY{p}{)} \PY{k}{for} \PY{n}{r} \PY{o+ow}{in} \PY{n}{r\PYZus{}samples}\PY{p}{]}

\PY{c+c1}{\PYZsh{} Plot the histogram of counts within r \PYZlt{} 0.8}
\PY{n}{bins\PYZus{}1} \PY{o}{=} \PY{n}{np}\PY{o}{.}\PY{n}{arange}\PY{p}{(}\PY{n+nb}{min}\PY{p}{(}\PY{n}{counts\PYZus{}1}\PY{p}{)}\PY{p}{,} \PY{n+nb}{max}\PY{p}{(}\PY{n}{counts\PYZus{}1}\PY{p}{)} \PY{o}{+} \PY{l+m+mi}{1}\PY{p}{)}
\PY{n}{plt}\PY{o}{.}\PY{n}{hist}\PY{p}{(}
    \PY{n}{counts\PYZus{}1}\PY{p}{,}
    \PY{n}{histtype} \PY{o}{=} \PY{l+s+s1}{\PYZsq{}}\PY{l+s+s1}{step}\PY{l+s+s1}{\PYZsq{}}\PY{p}{,}
    \PY{n}{density} \PY{o}{=} \PY{k+kc}{True}\PY{p}{,}
    \PY{n}{label} \PY{o}{=} \PY{l+s+s1}{\PYZsq{}}\PY{l+s+s1}{Simulated}\PY{l+s+s1}{\PYZsq{}}
\PY{p}{)}

\PY{c+c1}{\PYZsh{} Overlay Binomial and Poisson distributions}
\PY{c+c1}{\PYZsh{} NOTE: EDIT HERE to complete the Binomial parameters}
\PY{n}{pdf\PYZus{}binom\PYZus{}1} \PY{o}{=} \PY{n}{binom}\PY{o}{.}\PY{n}{pmf}\PY{p}{(}\PY{n}{bins\PYZus{}1}\PY{p}{,} \PY{n}{nstars}\PY{p}{,} \PY{n}{r\PYZus{}thres\PYZus{}1}\PY{o}{*}\PY{o}{*}\PY{l+m+mi}{3}\PY{p}{)}
\PY{c+c1}{\PYZsh{} NOTE: EDIT HERE to complete the Poisson parameter}
\PY{n}{pdf\PYZus{}poisson\PYZus{}1} \PY{o}{=} \PY{n}{poisson}\PY{o}{.}\PY{n}{pmf}\PY{p}{(}\PY{n}{bins\PYZus{}1}\PY{p}{,} \PY{n}{nstars} \PY{o}{*} \PY{n}{r\PYZus{}thres\PYZus{}1}\PY{o}{*}\PY{o}{*}\PY{l+m+mi}{3}\PY{p}{)}
\PY{n}{plt}\PY{o}{.}\PY{n}{plot}\PY{p}{(}\PY{n}{bins\PYZus{}1}\PY{p}{,} \PY{n}{pdf\PYZus{}binom\PYZus{}1}\PY{p}{,} \PY{l+s+s1}{\PYZsq{}}\PY{l+s+s1}{r\PYZhy{}}\PY{l+s+s1}{\PYZsq{}}\PY{p}{,} \PY{n}{label} \PY{o}{=} \PY{l+s+s1}{\PYZsq{}}\PY{l+s+s1}{Binomial}\PY{l+s+s1}{\PYZsq{}}\PY{p}{)}
\PY{n}{plt}\PY{o}{.}\PY{n}{plot}\PY{p}{(}\PY{n}{bins\PYZus{}1}\PY{p}{,} \PY{n}{pdf\PYZus{}poisson\PYZus{}1}\PY{p}{,} \PY{l+s+s1}{\PYZsq{}}\PY{l+s+s1}{k\PYZhy{}\PYZhy{}}\PY{l+s+s1}{\PYZsq{}}\PY{p}{,} \PY{n}{label} \PY{o}{=} \PY{l+s+s1}{\PYZsq{}}\PY{l+s+s1}{Poisson}\PY{l+s+s1}{\PYZsq{}}\PY{p}{)}

\PY{n}{plt}\PY{o}{.}\PY{n}{xlabel}\PY{p}{(}\PY{l+s+s1}{\PYZsq{}}\PY{l+s+s1}{Star Counts within \PYZdl{}r \PYZlt{} 0.8\PYZdl{}}\PY{l+s+s1}{\PYZsq{}}\PY{p}{)}
\PY{n}{plt}\PY{o}{.}\PY{n}{ylabel}\PY{p}{(}\PY{l+s+s1}{\PYZsq{}}\PY{l+s+s1}{Probability Density}\PY{l+s+s1}{\PYZsq{}}\PY{p}{)}
\PY{n}{plt}\PY{o}{.}\PY{n}{legend}\PY{p}{(}\PY{p}{)}
\PY{n}{plt}\PY{o}{.}\PY{n}{show}\PY{p}{(}\PY{p}{)}
\end{Verbatim}
\end{tcolorbox}

    \begin{center}
    \adjustimage{max size={0.9\linewidth}{0.9\paperheight}}{Exercise_week_03_files/Exercise_week_03_30_0.png}
    \end{center}
    { \hspace*{\fill} \\}
    
    \begin{tcolorbox}[breakable, size=fbox, boxrule=1pt, pad at break*=1mm,colback=cellbackground, colframe=cellborder]
\prompt{In}{incolor}{21}{\boxspacing}
\begin{Verbatim}[commandchars=\\\{\}]
\PY{c+c1}{\PYZsh{} Count stars within r \PYZlt{} 0.1}
\PY{n}{r\PYZus{}thres\PYZus{}2} \PY{o}{=} \PY{l+m+mf}{0.1}
\PY{n}{counts\PYZus{}2} \PY{o}{=} \PY{p}{[}\PY{n}{np}\PY{o}{.}\PY{n}{sum}\PY{p}{(}\PY{n}{r} \PY{o}{\PYZlt{}} \PY{n}{r\PYZus{}thres\PYZus{}2}\PY{p}{)} \PY{k}{for} \PY{n}{r} \PY{o+ow}{in} \PY{n}{r\PYZus{}samples}\PY{p}{]}

\PY{c+c1}{\PYZsh{} Plot the histogram of counts within r \PYZlt{} 0.1}
\PY{n}{bins\PYZus{}2} \PY{o}{=} \PY{n}{np}\PY{o}{.}\PY{n}{arange}\PY{p}{(}\PY{n+nb}{min}\PY{p}{(}\PY{n}{counts\PYZus{}2}\PY{p}{)}\PY{p}{,} \PY{n+nb}{max}\PY{p}{(}\PY{n}{counts\PYZus{}2}\PY{p}{)} \PY{o}{+} \PY{l+m+mi}{1}\PY{p}{)}
\PY{n}{plt}\PY{o}{.}\PY{n}{hist}\PY{p}{(}
    \PY{n}{counts\PYZus{}2}\PY{p}{,}
    \PY{n}{bins} \PY{o}{=} \PY{n}{bins\PYZus{}2}\PY{p}{,}
    \PY{n}{histtype} \PY{o}{=} \PY{l+s+s1}{\PYZsq{}}\PY{l+s+s1}{step}\PY{l+s+s1}{\PYZsq{}}\PY{p}{,}
    \PY{n}{density} \PY{o}{=} \PY{k+kc}{True}\PY{p}{,}
    \PY{n}{label} \PY{o}{=} \PY{l+s+s1}{\PYZsq{}}\PY{l+s+s1}{Simulated}\PY{l+s+s1}{\PYZsq{}}
\PY{p}{)}

\PY{c+c1}{\PYZsh{} Overlay Binomial and Poisson distributions}
\PY{c+c1}{\PYZsh{} NOTE: EDIT HERE to complete the Binomial parameters}
\PY{n}{pdf\PYZus{}binom\PYZus{}2} \PY{o}{=} \PY{n}{binom}\PY{o}{.}\PY{n}{pmf}\PY{p}{(}\PY{n}{bins\PYZus{}2}\PY{p}{,} \PY{n}{nstars}\PY{p}{,} \PY{n}{r\PYZus{}thres\PYZus{}2}\PY{o}{*}\PY{o}{*}\PY{l+m+mi}{3}\PY{p}{)}
\PY{c+c1}{\PYZsh{} NOTE: EDIT HERE to complete the Poisson parameter}
\PY{n}{pdf\PYZus{}poisson\PYZus{}2} \PY{o}{=} \PY{n}{poisson}\PY{o}{.}\PY{n}{pmf}\PY{p}{(}\PY{n}{bins\PYZus{}2}\PY{p}{,} \PY{n}{nstars} \PY{o}{*} \PY{n}{r\PYZus{}thres\PYZus{}2}\PY{o}{*}\PY{o}{*}\PY{l+m+mi}{3}\PY{p}{)}
\PY{n}{plt}\PY{o}{.}\PY{n}{step}\PY{p}{(}
    \PY{n}{bins\PYZus{}2}\PY{p}{,}
    \PY{n}{pdf\PYZus{}binom\PYZus{}2}\PY{p}{,}
    \PY{l+s+s1}{\PYZsq{}}\PY{l+s+s1}{r\PYZhy{}}\PY{l+s+s1}{\PYZsq{}}\PY{p}{,}
    \PY{n}{where} \PY{o}{=} \PY{l+s+s1}{\PYZsq{}}\PY{l+s+s1}{mid}\PY{l+s+s1}{\PYZsq{}}\PY{p}{,}
    \PY{n}{label} \PY{o}{=} \PY{l+s+s1}{\PYZsq{}}\PY{l+s+s1}{Binomial}\PY{l+s+s1}{\PYZsq{}}
\PY{p}{)}
\PY{n}{plt}\PY{o}{.}\PY{n}{step}\PY{p}{(}
    \PY{n}{bins\PYZus{}2}\PY{p}{,}
    \PY{n}{pdf\PYZus{}poisson\PYZus{}2}\PY{p}{,}
    \PY{l+s+s1}{\PYZsq{}}\PY{l+s+s1}{k\PYZhy{}\PYZhy{}}\PY{l+s+s1}{\PYZsq{}}\PY{p}{,}
    \PY{n}{where} \PY{o}{=} \PY{l+s+s1}{\PYZsq{}}\PY{l+s+s1}{mid}\PY{l+s+s1}{\PYZsq{}}\PY{p}{,}
    \PY{n}{label} \PY{o}{=} \PY{l+s+s1}{\PYZsq{}}\PY{l+s+s1}{Poisson}\PY{l+s+s1}{\PYZsq{}}
\PY{p}{)}

\PY{n}{plt}\PY{o}{.}\PY{n}{xlabel}\PY{p}{(}\PY{l+s+s1}{\PYZsq{}}\PY{l+s+s1}{Star Counts within \PYZdl{}r \PYZlt{} 0.1\PYZdl{}}\PY{l+s+s1}{\PYZsq{}}\PY{p}{)}
\PY{n}{plt}\PY{o}{.}\PY{n}{ylabel}\PY{p}{(}\PY{l+s+s1}{\PYZsq{}}\PY{l+s+s1}{Probability Density}\PY{l+s+s1}{\PYZsq{}}\PY{p}{)}
\PY{n}{plt}\PY{o}{.}\PY{n}{legend}\PY{p}{(}\PY{p}{)}
\PY{n}{plt}\PY{o}{.}\PY{n}{show}\PY{p}{(}\PY{p}{)}
\end{Verbatim}
\end{tcolorbox}

    \begin{center}
    \adjustimage{max size={0.9\linewidth}{0.9\paperheight}}{Exercise_week_03_files/Exercise_week_03_31_0.png}
    \end{center}
    { \hspace*{\fill} \\}
    
    \begin{tcolorbox}[breakable, size=fbox, boxrule=1pt, pad at break*=1mm,colback=cellbackground, colframe=cellborder]
\prompt{In}{incolor}{22}{\boxspacing}
\begin{Verbatim}[commandchars=\\\{\}]
\PY{c+c1}{\PYZsh{} Define spherical shells with equal thickness}
\PY{n}{r\PYZus{}edges} \PY{o}{=} \PY{n}{np}\PY{o}{.}\PY{n}{linspace}\PY{p}{(}\PY{l+m+mi}{0}\PY{p}{,} \PY{l+m+mi}{1}\PY{p}{,} \PY{l+m+mi}{11}\PY{p}{)} \PY{c+c1}{\PYZsh{} 10 shells}
\PY{n}{r\PYZus{}centers} \PY{o}{=} \PY{l+m+mf}{0.5} \PY{o}{*} \PY{p}{(}\PY{n}{r\PYZus{}edges}\PY{p}{[}\PY{p}{:}\PY{o}{\PYZhy{}}\PY{l+m+mi}{1}\PY{p}{]} \PY{o}{+} \PY{n}{r\PYZus{}edges}\PY{p}{[}\PY{l+m+mi}{1}\PY{p}{:}\PY{p}{]}\PY{p}{)}
\PY{n}{shell\PYZus{}volumes} \PY{o}{=} \PY{p}{(}\PY{l+m+mi}{4}\PY{o}{/}\PY{l+m+mi}{3}\PY{p}{)} \PY{o}{*} \PY{n}{np}\PY{o}{.}\PY{n}{pi} \PY{o}{*} \PY{p}{(}\PY{n}{r\PYZus{}edges}\PY{p}{[}\PY{l+m+mi}{1}\PY{p}{:}\PY{p}{]}\PY{o}{*}\PY{o}{*}\PY{l+m+mi}{3} \PY{o}{\PYZhy{}} \PY{n}{r\PYZus{}edges}\PY{p}{[}\PY{p}{:}\PY{o}{\PYZhy{}}\PY{l+m+mi}{1}\PY{p}{]}\PY{o}{*}\PY{o}{*}\PY{l+m+mi}{3}\PY{p}{)}
\PY{c+c1}{\PYZsh{} NOTE: EDIT HERE to compute the counts in each shell}
\PY{n}{shell\PYZus{}counts}\PY{p}{,} \PY{n}{\PYZus{}} \PY{o}{=} \PY{n}{np}\PY{o}{.}\PY{n}{histogram}\PY{p}{(}\PY{n}{r\PYZus{}sampled}\PY{p}{,} \PY{n}{bins} \PY{o}{=} \PY{n}{r\PYZus{}edges}\PY{p}{)}
\PY{c+c1}{\PYZsh{} NOTE: EDIT HERE to estimate the errors in counts}
\PY{c+c1}{\PYZsh{} Estimate errors from the ensemble of simulations (sample standard deviation across seeds)}
\PY{n}{counts\PYZus{}matrix} \PY{o}{=} \PY{n}{np}\PY{o}{.}\PY{n}{array}\PY{p}{(}\PY{p}{[}\PY{n}{np}\PY{o}{.}\PY{n}{histogram}\PY{p}{(}\PY{n}{r}\PY{p}{,} \PY{n}{bins} \PY{o}{=} \PY{n}{r\PYZus{}edges}\PY{p}{)}\PY{p}{[}\PY{l+m+mi}{0}\PY{p}{]} \PY{k}{for} \PY{n}{r} \PY{o+ow}{in} \PY{n}{r\PYZus{}samples}\PY{p}{]}\PY{p}{)}
\PY{n}{shell\PYZus{}count\PYZus{}errors} \PY{o}{=} \PY{n}{counts\PYZus{}matrix}\PY{o}{.}\PY{n}{std}\PY{p}{(}\PY{n}{axis} \PY{o}{=} \PY{l+m+mi}{0}\PY{p}{,} \PY{n}{ddof} \PY{o}{=} \PY{l+m+mi}{1}\PY{p}{)}
\PY{n}{shell\PYZus{}densities} \PY{o}{=} \PY{n}{shell\PYZus{}counts} \PY{o}{/} \PY{n}{shell\PYZus{}volumes}
\PY{n}{shell\PYZus{}density\PYZus{}errors} \PY{o}{=} \PY{n}{shell\PYZus{}count\PYZus{}errors} \PY{o}{/} \PY{n}{shell\PYZus{}volumes}

\PY{c+c1}{\PYZsh{} Replot the stellar density distribution with error bars}
\PY{n}{plt}\PY{o}{.}\PY{n}{step}\PY{p}{(}
    \PY{n}{r\PYZus{}centers}\PY{p}{,}
    \PY{n}{shell\PYZus{}densities}\PY{p}{,}
    \PY{n}{where} \PY{o}{=} \PY{l+s+s1}{\PYZsq{}}\PY{l+s+s1}{mid}\PY{l+s+s1}{\PYZsq{}}\PY{p}{,}
    \PY{n}{label} \PY{o}{=} \PY{l+s+s1}{\PYZsq{}}\PY{l+s+s1}{Simulated}\PY{l+s+s1}{\PYZsq{}}
\PY{p}{)}
\PY{n}{plt}\PY{o}{.}\PY{n}{errorbar}\PY{p}{(}
    \PY{n}{r\PYZus{}centers}\PY{p}{,}
    \PY{n}{shell\PYZus{}densities}\PY{p}{,}
    \PY{n}{yerr} \PY{o}{=} \PY{n}{shell\PYZus{}density\PYZus{}errors}\PY{p}{,}
    \PY{n}{fmt} \PY{o}{=} \PY{l+s+s1}{\PYZsq{}}\PY{l+s+s1}{.}\PY{l+s+s1}{\PYZsq{}}\PY{p}{,}
    \PY{n}{ls} \PY{o}{=} \PY{l+s+s1}{\PYZsq{}}\PY{l+s+s1}{\PYZhy{}}\PY{l+s+s1}{\PYZsq{}}\PY{p}{,}
    \PY{n}{capsize} \PY{o}{=} \PY{l+m+mi}{3}\PY{p}{,}
    \PY{n}{color} \PY{o}{=} \PY{l+s+s1}{\PYZsq{}}\PY{l+s+s1}{C0}\PY{l+s+s1}{\PYZsq{}}\PY{p}{,}
    \PY{n}{label} \PY{o}{=} \PY{l+s+s1}{\PYZsq{}}\PY{l+s+s1}{With Error Bars}\PY{l+s+s1}{\PYZsq{}}
\PY{p}{)}

\PY{c+c1}{\PYZsh{} Overlay the theoretical constant density}
\PY{n}{density\PYZus{}theory} \PY{o}{=} \PY{n}{nstars} \PY{o}{/} \PY{p}{(}\PY{p}{(}\PY{l+m+mf}{4.0} \PY{o}{/} \PY{l+m+mf}{3.0}\PY{p}{)} \PY{o}{*} \PY{n}{np}\PY{o}{.}\PY{n}{pi}\PY{p}{)}
\PY{n}{plt}\PY{o}{.}\PY{n}{axhline}\PY{p}{(}
    \PY{n}{density\PYZus{}theory}\PY{p}{,}
    \PY{n}{color} \PY{o}{=} \PY{l+s+s1}{\PYZsq{}}\PY{l+s+s1}{r}\PY{l+s+s1}{\PYZsq{}}\PY{p}{,}
    \PY{n}{linestyle} \PY{o}{=} \PY{l+s+s1}{\PYZsq{}}\PY{l+s+s1}{\PYZhy{}\PYZhy{}}\PY{l+s+s1}{\PYZsq{}}\PY{p}{,}
    \PY{n}{label} \PY{o}{=} \PY{l+s+s1}{\PYZsq{}}\PY{l+s+s1}{Theory}\PY{l+s+s1}{\PYZsq{}}
\PY{p}{)}

\PY{n}{plt}\PY{o}{.}\PY{n}{xlabel}\PY{p}{(}\PY{l+s+s1}{\PYZsq{}}\PY{l+s+s1}{Radius \PYZdl{}r\PYZdl{}}\PY{l+s+s1}{\PYZsq{}}\PY{p}{)}
\PY{n}{plt}\PY{o}{.}\PY{n}{ylabel}\PY{p}{(}\PY{l+s+s1}{\PYZsq{}}\PY{l+s+s1}{Stellar Density}\PY{l+s+s1}{\PYZsq{}}\PY{p}{)}
\PY{n}{plt}\PY{o}{.}\PY{n}{legend}\PY{p}{(}\PY{p}{)}
\PY{n}{plt}\PY{o}{.}\PY{n}{show}\PY{p}{(}\PY{p}{)}
\end{Verbatim}
\end{tcolorbox}

    \begin{center}
    \adjustimage{max size={0.9\linewidth}{0.9\paperheight}}{Exercise_week_03_files/Exercise_week_03_32_0.png}
    \end{center}
    { \hspace*{\fill} \\}
    
    \subparagraph{Discussions}\label{discussions}

可以看出在 \(r\) 更小的区域,星体分布的概率更小,这是符合实际的 (\(r\)
越小,体积越小). 另外,分区域的星体数量分布近似于 Poisson 分布和 Gauss
分布.

    \paragraph{Question 3.3: Flux
distribution}\label{question-3.3-flux-distribution}

Given the luminosity of a star \(L\) (energy per unit time), its
observed flux at distance \(r\) follows the inverse-square law:

\[F = \frac{L}{4 \pi r^2}\]

where \(F\) is the flux (energy per unit area per unit time).

\textbf{Tasks:} 1. Using the \(N=10,000\) stars from Question 3.1, draw
luminosities in 4 ways: - All stars have the same luminosity \(L=10\). -
A uniform luminosity distribution in the range \(L \in [0, 20)\). - A
log-normal distribution with parameters \(\mu=2.2\) and \(\sigma=0.3\).
- A power-law (Pareto) distribution with index \(\alpha=2\) and minimum
luminosity \(L_{\rm min}=1\). 2. For each luminosity distribution (i.e.,
luminosity function) except the first, plot the histogram of simulated
\(L\) and overlay the corresponding theoretical distributions. 3. For
each luminosity function, plot the histogram of observed stellar fluxes
with appropriate axes scales. Can you guess the underlying flux
distribution function and overlay it on the histogram?

    \begin{tcolorbox}[breakable, size=fbox, boxrule=1pt, pad at break*=1mm,colback=cellbackground, colframe=cellborder]
\prompt{In}{incolor}{23}{\boxspacing}
\begin{Verbatim}[commandchars=\\\{\}]
\PY{n}{rng} \PY{o}{=} \PY{n}{np}\PY{o}{.}\PY{n}{random}\PY{o}{.}\PY{n}{default\PYZus{}rng}\PY{p}{(}\PY{n}{seed} \PY{o}{+} \PY{l+m+mi}{1}\PY{p}{)}

\PY{n}{luminosity\PYZus{}1} \PY{o}{=} \PY{n}{np}\PY{o}{.}\PY{n}{ones}\PY{p}{(}\PY{n}{nstars}\PY{p}{)} \PY{o}{*} \PY{l+m+mi}{10} \PY{c+c1}{\PYZsh{} Luminosity of 10 for all stars}
\PY{n}{luminosity\PYZus{}2} \PY{o}{=} \PY{n}{rng}\PY{o}{.}\PY{n}{uniform}\PY{p}{(}\PY{l+m+mi}{0}\PY{p}{,} \PY{l+m+mi}{20}\PY{p}{,} \PY{n}{nstars}\PY{p}{)} \PY{c+c1}{\PYZsh{} Uniform distribution}
\PY{n}{luminosity\PYZus{}3} \PY{o}{=} \PY{n}{rng}\PY{o}{.}\PY{n}{lognormal}\PY{p}{(}
    \PY{n}{mean} \PY{o}{=} \PY{l+m+mf}{2.2}\PY{p}{,} 
    \PY{n}{sigma} \PY{o}{=} \PY{l+m+mf}{0.3}\PY{p}{,} 
    \PY{n}{size} \PY{o}{=} \PY{n}{nstars}
\PY{p}{)} \PY{c+c1}{\PYZsh{} Log\PYZhy{}normal distribution}
\PY{n}{luminosity\PYZus{}4} \PY{o}{=} \PY{p}{(}\PY{n}{rng}\PY{o}{.}\PY{n}{pareto}\PY{p}{(}\PY{l+m+mi}{2}\PY{p}{,} \PY{n}{nstars}\PY{p}{)} \PY{o}{+} \PY{l+m+mi}{1}\PY{p}{)} \PY{c+c1}{\PYZsh{} Power\PYZhy{}law distribution}
\end{Verbatim}
\end{tcolorbox}

    \begin{tcolorbox}[breakable, size=fbox, boxrule=1pt, pad at break*=1mm,colback=cellbackground, colframe=cellborder]
\prompt{In}{incolor}{24}{\boxspacing}
\begin{Verbatim}[commandchars=\\\{\}]
\PY{c+c1}{\PYZsh{} Plot the histograms of luminosities and overlay theoretical distributions}
\PY{n}{fig} \PY{o}{=} \PY{n}{plt}\PY{o}{.}\PY{n}{figure}\PY{p}{(}\PY{n}{figsize} \PY{o}{=} \PY{p}{(}\PY{l+m+mi}{12}\PY{p}{,} \PY{l+m+mi}{4}\PY{p}{)}\PY{p}{)}
\PY{n}{ax} \PY{o}{=} \PY{n}{fig}\PY{o}{.}\PY{n}{subplots}\PY{p}{(}\PY{l+m+mi}{1}\PY{p}{,} \PY{l+m+mi}{3}\PY{p}{)}\PY{o}{.}\PY{n}{flatten}\PY{p}{(}\PY{p}{)}

\PY{c+c1}{\PYZsh{} Uniform distribution}
\PY{n}{ax}\PY{p}{[}\PY{l+m+mi}{0}\PY{p}{]}\PY{o}{.}\PY{n}{hist}\PY{p}{(}
    \PY{n}{luminosity\PYZus{}2}\PY{p}{,} 
    \PY{n}{histtype} \PY{o}{=} \PY{l+s+s1}{\PYZsq{}}\PY{l+s+s1}{step}\PY{l+s+s1}{\PYZsq{}}\PY{p}{,}
    \PY{n}{density} \PY{o}{=} \PY{k+kc}{True}\PY{p}{,}
    \PY{n}{label} \PY{o}{=} \PY{l+s+s1}{\PYZsq{}}\PY{l+s+s1}{Simulated}\PY{l+s+s1}{\PYZsq{}}
\PY{p}{)}
\PY{n}{x\PYZus{}uniform} \PY{o}{=} \PY{n}{np}\PY{o}{.}\PY{n}{linspace}\PY{p}{(}\PY{l+m+mi}{0}\PY{p}{,} \PY{l+m+mi}{20}\PY{p}{,} \PY{l+m+mi}{100}\PY{p}{)}
\PY{c+c1}{\PYZsh{} NOTE: EDIT HERE to compute the theoretical PDF of the uniform distribution}
\PY{n}{y\PYZus{}uniform} \PY{o}{=} \PY{n}{np}\PY{o}{.}\PY{n}{where}\PY{p}{(}\PY{p}{(}\PY{n}{x\PYZus{}uniform} \PY{o}{\PYZgt{}}\PY{o}{=} \PY{l+m+mi}{0}\PY{p}{)} \PY{o}{\PYZam{}} \PY{p}{(}\PY{n}{x\PYZus{}uniform} \PY{o}{\PYZlt{}} \PY{l+m+mi}{20}\PY{p}{)}\PY{p}{,} \PY{l+m+mf}{1.0} \PY{o}{/} \PY{l+m+mf}{20.0}\PY{p}{,} \PY{l+m+mf}{0.0}\PY{p}{)}
\PY{n}{ax}\PY{p}{[}\PY{l+m+mi}{0}\PY{p}{]}\PY{o}{.}\PY{n}{plot}\PY{p}{(}\PY{n}{x\PYZus{}uniform}\PY{p}{,} \PY{n}{y\PYZus{}uniform}\PY{p}{,} \PY{l+s+s1}{\PYZsq{}}\PY{l+s+s1}{r\PYZhy{}}\PY{l+s+s1}{\PYZsq{}}\PY{p}{,} \PY{n}{label} \PY{o}{=} \PY{l+s+s1}{\PYZsq{}}\PY{l+s+s1}{Theory}\PY{l+s+s1}{\PYZsq{}}\PY{p}{)}
\PY{n}{ax}\PY{p}{[}\PY{l+m+mi}{0}\PY{p}{]}\PY{o}{.}\PY{n}{set\PYZus{}title}\PY{p}{(}\PY{l+s+s1}{\PYZsq{}}\PY{l+s+s1}{Uniform Distribution}\PY{l+s+s1}{\PYZsq{}}\PY{p}{)}

\PY{c+c1}{\PYZsh{} Log\PYZhy{}normal distribution}
\PY{n}{ax}\PY{p}{[}\PY{l+m+mi}{1}\PY{p}{]}\PY{o}{.}\PY{n}{hist}\PY{p}{(}
    \PY{n}{luminosity\PYZus{}3}\PY{p}{,}
    \PY{n}{histtype} \PY{o}{=} \PY{l+s+s1}{\PYZsq{}}\PY{l+s+s1}{step}\PY{l+s+s1}{\PYZsq{}}\PY{p}{,}
    \PY{n}{density} \PY{o}{=} \PY{k+kc}{True}\PY{p}{,}
    \PY{n}{label} \PY{o}{=} \PY{l+s+s1}{\PYZsq{}}\PY{l+s+s1}{Simulated}\PY{l+s+s1}{\PYZsq{}}
\PY{p}{)}
\PY{n}{x\PYZus{}lognormal} \PY{o}{=} \PY{n}{np}\PY{o}{.}\PY{n}{linspace}\PY{p}{(}\PY{l+m+mf}{0.1}\PY{p}{,} \PY{l+m+mi}{25}\PY{p}{,} \PY{l+m+mi}{100}\PY{p}{)}
\PY{c+c1}{\PYZsh{} NOTE: EDIT HERE to compute the theoretical PDF of the log\PYZhy{}normal distribution}
\PY{n}{mu\PYZus{}ln} \PY{o}{=} \PY{l+m+mf}{2.2}
\PY{n}{sigma\PYZus{}ln} \PY{o}{=} \PY{l+m+mf}{0.3}
\PY{n}{y\PYZus{}lognormal} \PY{o}{=} \PY{p}{(}\PY{l+m+mf}{1.0} \PY{o}{/} \PY{p}{(}\PY{n}{x\PYZus{}lognormal} \PY{o}{*} \PY{n}{sigma\PYZus{}ln} \PY{o}{*} \PY{n}{np}\PY{o}{.}\PY{n}{sqrt}\PY{p}{(}\PY{l+m+mi}{2} \PY{o}{*} \PY{n}{np}\PY{o}{.}\PY{n}{pi}\PY{p}{)}\PY{p}{)}\PY{p}{)} \PY{o}{*} \PY{n}{np}\PY{o}{.}\PY{n}{exp}\PY{p}{(}\PY{o}{\PYZhy{}} \PY{p}{(}\PY{n}{np}\PY{o}{.}\PY{n}{log}\PY{p}{(}\PY{n}{x\PYZus{}lognormal}\PY{p}{)} \PY{o}{\PYZhy{}} \PY{n}{mu\PYZus{}ln}\PY{p}{)}\PY{o}{*}\PY{o}{*}\PY{l+m+mi}{2} \PY{o}{/} \PY{p}{(}\PY{l+m+mi}{2} \PY{o}{*} \PY{n}{sigma\PYZus{}ln}\PY{o}{*}\PY{o}{*}\PY{l+m+mi}{2}\PY{p}{)}\PY{p}{)}
\PY{n}{ax}\PY{p}{[}\PY{l+m+mi}{1}\PY{p}{]}\PY{o}{.}\PY{n}{plot}\PY{p}{(}\PY{n}{x\PYZus{}lognormal}\PY{p}{,} \PY{n}{y\PYZus{}lognormal}\PY{p}{,} \PY{l+s+s1}{\PYZsq{}}\PY{l+s+s1}{r\PYZhy{}}\PY{l+s+s1}{\PYZsq{}}\PY{p}{,} \PY{n}{label} \PY{o}{=} \PY{l+s+s1}{\PYZsq{}}\PY{l+s+s1}{Theory}\PY{l+s+s1}{\PYZsq{}}\PY{p}{)}
\PY{n}{ax}\PY{p}{[}\PY{l+m+mi}{1}\PY{p}{]}\PY{o}{.}\PY{n}{set\PYZus{}title}\PY{p}{(}\PY{l+s+s1}{\PYZsq{}}\PY{l+s+s1}{Log\PYZhy{}normal Distribution}\PY{l+s+s1}{\PYZsq{}}\PY{p}{)}

\PY{c+c1}{\PYZsh{} Power\PYZhy{}law distribution}
\PY{n}{bins} \PY{o}{=} \PY{n}{np}\PY{o}{.}\PY{n}{logspace}\PY{p}{(}\PY{l+m+mi}{0}\PY{p}{,} \PY{l+m+mi}{2}\PY{p}{,} \PY{l+m+mi}{20}\PY{p}{)}
\PY{n}{ax}\PY{p}{[}\PY{l+m+mi}{2}\PY{p}{]}\PY{o}{.}\PY{n}{hist}\PY{p}{(}
    \PY{n}{luminosity\PYZus{}4}\PY{p}{,}
    \PY{n}{bins} \PY{o}{=} \PY{n}{bins}\PY{p}{,}
    \PY{n}{histtype} \PY{o}{=} \PY{l+s+s1}{\PYZsq{}}\PY{l+s+s1}{step}\PY{l+s+s1}{\PYZsq{}}\PY{p}{,}
    \PY{n}{density} \PY{o}{=} \PY{k+kc}{True}\PY{p}{,}
    \PY{n}{label} \PY{o}{=} \PY{l+s+s1}{\PYZsq{}}\PY{l+s+s1}{Simulated}\PY{l+s+s1}{\PYZsq{}}
\PY{p}{)}
\PY{n}{x\PYZus{}powerlaw} \PY{o}{=} \PY{n}{np}\PY{o}{.}\PY{n}{linspace}\PY{p}{(}\PY{l+m+mi}{1}\PY{p}{,} \PY{l+m+mi}{100}\PY{p}{,} \PY{l+m+mi}{100}\PY{p}{)}
\PY{c+c1}{\PYZsh{} NOTE: EDIT HERE to compute the theoretical PDF of the power\PYZhy{}law distribution}
\PY{n}{y\PYZus{}powerlaw} \PY{o}{=} \PY{n}{np}\PY{o}{.}\PY{n}{where}\PY{p}{(}\PY{n}{x\PYZus{}powerlaw} \PY{o}{\PYZgt{}}\PY{o}{=} \PY{l+m+mi}{1}\PY{p}{,} \PY{l+m+mf}{2.0} \PY{o}{*} \PY{n}{x\PYZus{}powerlaw}\PY{o}{*}\PY{o}{*}\PY{p}{(}\PY{o}{\PYZhy{}}\PY{l+m+mi}{3}\PY{p}{)}\PY{p}{,} \PY{l+m+mf}{0.0}\PY{p}{)}
\PY{n}{ax}\PY{p}{[}\PY{l+m+mi}{2}\PY{p}{]}\PY{o}{.}\PY{n}{plot}\PY{p}{(}\PY{n}{x\PYZus{}powerlaw}\PY{p}{,} \PY{n}{y\PYZus{}powerlaw}\PY{p}{,} \PY{l+s+s1}{\PYZsq{}}\PY{l+s+s1}{r\PYZhy{}}\PY{l+s+s1}{\PYZsq{}}\PY{p}{,} \PY{n}{label} \PY{o}{=} \PY{l+s+s1}{\PYZsq{}}\PY{l+s+s1}{Theory}\PY{l+s+s1}{\PYZsq{}}\PY{p}{)}
\PY{n}{ax}\PY{p}{[}\PY{l+m+mi}{2}\PY{p}{]}\PY{o}{.}\PY{n}{set\PYZus{}title}\PY{p}{(}\PY{l+s+s1}{\PYZsq{}}\PY{l+s+s1}{Power\PYZhy{}law Distribution}\PY{l+s+s1}{\PYZsq{}}\PY{p}{)}
\PY{n}{ax}\PY{p}{[}\PY{l+m+mi}{2}\PY{p}{]}\PY{o}{.}\PY{n}{set\PYZus{}xscale}\PY{p}{(}\PY{l+s+s1}{\PYZsq{}}\PY{l+s+s1}{log}\PY{l+s+s1}{\PYZsq{}}\PY{p}{)}
\PY{n}{ax}\PY{p}{[}\PY{l+m+mi}{2}\PY{p}{]}\PY{o}{.}\PY{n}{set\PYZus{}yscale}\PY{p}{(}\PY{l+s+s1}{\PYZsq{}}\PY{l+s+s1}{log}\PY{l+s+s1}{\PYZsq{}}\PY{p}{)}

\PY{k}{for} \PY{n}{a} \PY{o+ow}{in} \PY{n}{ax}\PY{p}{:}
    \PY{n}{a}\PY{o}{.}\PY{n}{set\PYZus{}xlabel}\PY{p}{(}\PY{l+s+s1}{\PYZsq{}}\PY{l+s+s1}{Luminosity}\PY{l+s+s1}{\PYZsq{}}\PY{p}{)}
    \PY{n}{a}\PY{o}{.}\PY{n}{set\PYZus{}ylabel}\PY{p}{(}\PY{l+s+s1}{\PYZsq{}}\PY{l+s+s1}{Probability Density}\PY{l+s+s1}{\PYZsq{}}\PY{p}{)}

\PY{n}{plt}\PY{o}{.}\PY{n}{legend}\PY{p}{(}\PY{p}{)}
\PY{n}{plt}\PY{o}{.}\PY{n}{tight\PYZus{}layout}\PY{p}{(}\PY{p}{)}
\PY{n}{plt}\PY{o}{.}\PY{n}{show}\PY{p}{(}\PY{p}{)}
\end{Verbatim}
\end{tcolorbox}

    \begin{center}
    \adjustimage{max size={0.9\linewidth}{0.9\paperheight}}{Exercise_week_03_files/Exercise_week_03_36_0.png}
    \end{center}
    { \hspace*{\fill} \\}
    
    理论分布如何计算?实际上我们这里有几个分布:对于每个 \(r\),\(L\)
有一个具体的分布,但是这个分布可能和 \(r\) 有关;\(r\)
自身是均匀分布;我们要求的是 \(F\) 的分布函数.

考虑处在 \(r\to r+\text{d}r\) 段的 \(r\),概率为
\(P_r(r)\text{d}r\),这和在 \(f\to f+\text{d}f\) 段的 \(F\)
概率相等,于是

\[
P_r(r)\text{d}r = P_f(f)\text{d}f \Longrightarrow P_f(f) = P_r(r)\left|\frac{\text{d}r}{\text{d}f}\right|
\]

但是,这里的 \(f\) 和 \(P_f(f)\) 并不是 \(F\) 和
\(P_F(F)\)!因为我们并未考虑 \(L\) 自己分布的影响,所以上面得到的其实是
\(P_F(F|L = L_0)\). 利用乘法规则,积分一次应该能够得到我们想要的结果:

\[
P_F(F) = \int_{L'} P_F(F|L = L')P(L')\text{d}L'
\]

    \begin{tcolorbox}[breakable, size=fbox, boxrule=1pt, pad at break*=1mm,colback=cellbackground, colframe=cellborder]
\prompt{In}{incolor}{25}{\boxspacing}
\begin{Verbatim}[commandchars=\\\{\}]
\PY{c+c1}{\PYZsh{} Compute the observed fluxes}
\PY{n}{luminosities} \PY{o}{=} \PY{p}{[}\PY{n}{luminosity\PYZus{}1}\PY{p}{,} \PY{n}{luminosity\PYZus{}2}\PY{p}{,} \PY{n}{luminosity\PYZus{}3}\PY{p}{,} \PY{n}{luminosity\PYZus{}4}\PY{p}{]}
\PY{n}{labels} \PY{o}{=} \PY{p}{[}\PY{l+s+s1}{\PYZsq{}}\PY{l+s+s1}{Luminosity = 10}\PY{l+s+s1}{\PYZsq{}}\PY{p}{,} \PY{l+s+s1}{\PYZsq{}}\PY{l+s+s1}{Uniform}\PY{l+s+s1}{\PYZsq{}}\PY{p}{,} \PY{l+s+s1}{\PYZsq{}}\PY{l+s+s1}{Log\PYZhy{}normal}\PY{l+s+s1}{\PYZsq{}}\PY{p}{,} \PY{l+s+s1}{\PYZsq{}}\PY{l+s+s1}{Power\PYZhy{}law}\PY{l+s+s1}{\PYZsq{}}\PY{p}{]}
\PY{n}{fluxes} \PY{o}{=} \PY{p}{[}\PY{n}{L} \PY{o}{/} \PY{p}{(}\PY{l+m+mi}{4} \PY{o}{*} \PY{n}{np}\PY{o}{.}\PY{n}{pi} \PY{o}{*} \PY{n}{r\PYZus{}sampled}\PY{o}{*}\PY{o}{*}\PY{l+m+mi}{2}\PY{p}{)} \PY{k}{for} \PY{n}{L} \PY{o+ow}{in} \PY{n}{luminosities}\PY{p}{]}

\PY{c+c1}{\PYZsh{} Plot the histograms of observed fluxes in the same panel and overlay theoretical distributions}
\PY{n}{fig} \PY{o}{=} \PY{n}{plt}\PY{o}{.}\PY{n}{figure}\PY{p}{(}\PY{n}{figsize}\PY{o}{=}\PY{p}{(}\PY{l+m+mi}{8}\PY{p}{,} \PY{l+m+mi}{8}\PY{p}{)}\PY{p}{)}
\PY{n}{ax} \PY{o}{=} \PY{n}{fig}\PY{o}{.}\PY{n}{subplots}\PY{p}{(}\PY{l+m+mi}{2}\PY{p}{,} \PY{l+m+mi}{2}\PY{p}{)}\PY{o}{.}\PY{n}{flatten}\PY{p}{(}\PY{p}{)}

\PY{c+c1}{\PYZsh{} Plot the histograms of observed fluxes}
\PY{c+c1}{\PYZsh{} For clarity set x\PYZhy{}scale to log for these flux histograms}
\PY{k}{for} \PY{n}{i} \PY{o+ow}{in} \PY{n+nb}{range}\PY{p}{(}\PY{l+m+mi}{4}\PY{p}{)}\PY{p}{:}
    \PY{n}{ax}\PY{p}{[}\PY{n}{i}\PY{p}{]}\PY{o}{.}\PY{n}{hist}\PY{p}{(}
        \PY{n}{fluxes}\PY{p}{[}\PY{n}{i}\PY{p}{]}\PY{p}{,}
        \PY{n}{bins} \PY{o}{=} \PY{l+m+mi}{100}\PY{p}{,}
        \PY{n}{density} \PY{o}{=} \PY{k+kc}{True}\PY{p}{,}
        \PY{n}{histtype} \PY{o}{=} \PY{l+s+s1}{\PYZsq{}}\PY{l+s+s1}{step}\PY{l+s+s1}{\PYZsq{}}\PY{p}{,}
        \PY{n}{label} \PY{o}{=} \PY{n}{labels}\PY{p}{[}\PY{n}{i}\PY{p}{]}\PY{p}{,}
    \PY{p}{)}

\PY{k+kn}{from}\PY{+w}{ }\PY{n+nn}{math}\PY{+w}{ }\PY{k+kn}{import} \PY{n}{pi}
\PY{k+kn}{from}\PY{+w}{ }\PY{n+nn}{scipy}\PY{n+nn}{.}\PY{n+nn}{special}\PY{+w}{ }\PY{k+kn}{import} \PY{n}{erf}

\PY{c+c1}{\PYZsh{} numerical prefactor for the analytic flux PDFs}
\PY{n}{\PYZus{}PREF} \PY{o}{=} \PY{p}{(}\PY{l+m+mf}{3.0} \PY{o}{/} \PY{l+m+mf}{2.0}\PY{p}{)} \PY{o}{/} \PY{p}{(}\PY{p}{(}\PY{l+m+mf}{4.0} \PY{o}{*} \PY{n}{pi}\PY{p}{)} \PY{o}{*}\PY{o}{*} \PY{l+m+mf}{1.5}\PY{p}{)}

\PY{k}{def}\PY{+w}{ }\PY{n+nf}{pdf\PYZus{}flux\PYZus{}const}\PY{p}{(}\PY{n}{f}\PY{p}{,} \PY{n}{L0}\PY{p}{)}\PY{p}{:}
    \PY{n}{f} \PY{o}{=} \PY{n}{np}\PY{o}{.}\PY{n}{asarray}\PY{p}{(}\PY{n}{f}\PY{p}{,} \PY{n}{dtype}\PY{o}{=}\PY{n+nb}{float}\PY{p}{)}
    \PY{n}{fmin} \PY{o}{=} \PY{n}{L0} \PY{o}{/} \PY{p}{(}\PY{l+m+mf}{4.0} \PY{o}{*} \PY{n}{pi}\PY{p}{)}
    \PY{n}{p} \PY{o}{=} \PY{n}{np}\PY{o}{.}\PY{n}{zeros\PYZus{}like}\PY{p}{(}\PY{n}{f}\PY{p}{)}
    \PY{n}{mask} \PY{o}{=} \PY{n}{f} \PY{o}{\PYZgt{}}\PY{o}{=} \PY{n}{fmin}
    \PY{n}{p}\PY{p}{[}\PY{n}{mask}\PY{p}{]} \PY{o}{=} \PY{n}{\PYZus{}PREF} \PY{o}{*} \PY{p}{(}\PY{n}{f}\PY{p}{[}\PY{n}{mask}\PY{p}{]} \PY{o}{*}\PY{o}{*} \PY{o}{\PYZhy{}}\PY{l+m+mf}{2.5}\PY{p}{)} \PY{o}{*} \PY{p}{(}\PY{n}{L0} \PY{o}{*}\PY{o}{*} \PY{l+m+mf}{1.5}\PY{p}{)}
    \PY{k}{return} \PY{n}{p}

\PY{k}{def}\PY{+w}{ }\PY{n+nf}{pdf\PYZus{}flux\PYZus{}uniform}\PY{p}{(}\PY{n}{f}\PY{p}{,} \PY{n}{low}\PY{p}{,} \PY{n}{high}\PY{p}{)}\PY{p}{:}
    \PY{n}{f} \PY{o}{=} \PY{n}{np}\PY{o}{.}\PY{n}{asarray}\PY{p}{(}\PY{n}{f}\PY{p}{,} \PY{n}{dtype}\PY{o}{=}\PY{n+nb}{float}\PY{p}{)}
    \PY{n}{a} \PY{o}{=} \PY{l+m+mf}{4.0} \PY{o}{*} \PY{n}{pi} \PY{o}{*} \PY{n}{f}
    \PY{n}{width} \PY{o}{=} \PY{n}{high} \PY{o}{\PYZhy{}} \PY{n}{low}
    \PY{n}{I} \PY{o}{=} \PY{n}{np}\PY{o}{.}\PY{n}{zeros\PYZus{}like}\PY{p}{(}\PY{n}{f}\PY{p}{)}
    \PY{n}{mask} \PY{o}{=} \PY{n}{a} \PY{o}{\PYZgt{}} \PY{n}{low}
    \PY{n}{a\PYZus{}clip} \PY{o}{=} \PY{n}{np}\PY{o}{.}\PY{n}{minimum}\PY{p}{(}\PY{n}{a}\PY{p}{,} \PY{n}{high}\PY{p}{)}
    \PY{n}{I}\PY{p}{[}\PY{n}{mask}\PY{p}{]} \PY{o}{=} \PY{p}{(}\PY{p}{(}\PY{l+m+mf}{2.0} \PY{o}{/} \PY{l+m+mf}{5.0}\PY{p}{)} \PY{o}{*} \PY{p}{(}\PY{n}{a\PYZus{}clip}\PY{p}{[}\PY{n}{mask}\PY{p}{]} \PY{o}{*}\PY{o}{*} \PY{l+m+mf}{2.5} \PY{o}{\PYZhy{}} \PY{n}{low} \PY{o}{*}\PY{o}{*} \PY{l+m+mf}{2.5}\PY{p}{)}\PY{p}{)} \PY{o}{/} \PY{n}{width}
    \PY{k}{return} \PY{n}{\PYZus{}PREF} \PY{o}{*} \PY{n}{np}\PY{o}{.}\PY{n}{where}\PY{p}{(}\PY{n}{f} \PY{o}{\PYZgt{}} \PY{l+m+mi}{0}\PY{p}{,} \PY{n}{f} \PY{o}{*}\PY{o}{*} \PY{o}{\PYZhy{}}\PY{l+m+mf}{2.5}\PY{p}{,} \PY{l+m+mf}{0.0}\PY{p}{)} \PY{o}{*} \PY{n}{I}

\PY{k}{def}\PY{+w}{ }\PY{n+nf}{\PYZus{}Phi}\PY{p}{(}\PY{n}{z}\PY{p}{)}\PY{p}{:}
    \PY{c+c1}{\PYZsh{} Standard normal CDF using erf}
    \PY{k}{return} \PY{l+m+mf}{0.5} \PY{o}{*} \PY{p}{(}\PY{l+m+mf}{1.0} \PY{o}{+} \PY{n}{erf}\PY{p}{(}\PY{n}{z} \PY{o}{/} \PY{n}{np}\PY{o}{.}\PY{n}{sqrt}\PY{p}{(}\PY{l+m+mf}{2.0}\PY{p}{)}\PY{p}{)}\PY{p}{)}

\PY{k}{def}\PY{+w}{ }\PY{n+nf}{pdf\PYZus{}flux\PYZus{}lognormal}\PY{p}{(}\PY{n}{f}\PY{p}{,} \PY{n}{mu}\PY{p}{,} \PY{n}{sigma}\PY{p}{)}\PY{p}{:}
    \PY{n}{f} \PY{o}{=} \PY{n}{np}\PY{o}{.}\PY{n}{asarray}\PY{p}{(}\PY{n}{f}\PY{p}{,} \PY{n}{dtype}\PY{o}{=}\PY{n+nb}{float}\PY{p}{)}
    \PY{n}{a} \PY{o}{=} \PY{l+m+mf}{4.0} \PY{o}{*} \PY{n}{pi} \PY{o}{*} \PY{n}{np}\PY{o}{.}\PY{n}{maximum}\PY{p}{(}\PY{n}{f}\PY{p}{,} \PY{l+m+mf}{0.0}\PY{p}{)}
    \PY{n}{k} \PY{o}{=} \PY{l+m+mf}{1.5}
    \PY{n}{pref\PYZus{}I} \PY{o}{=} \PY{n}{np}\PY{o}{.}\PY{n}{exp}\PY{p}{(}\PY{n}{k} \PY{o}{*} \PY{n}{mu} \PY{o}{+} \PY{l+m+mf}{0.5} \PY{o}{*} \PY{p}{(}\PY{n}{k} \PY{o}{*}\PY{o}{*} \PY{l+m+mi}{2}\PY{p}{)} \PY{o}{*} \PY{p}{(}\PY{n}{sigma} \PY{o}{*}\PY{o}{*} \PY{l+m+mi}{2}\PY{p}{)}\PY{p}{)}
    \PY{n}{z} \PY{o}{=} \PY{p}{(}\PY{n}{np}\PY{o}{.}\PY{n}{log}\PY{p}{(}\PY{n}{np}\PY{o}{.}\PY{n}{where}\PY{p}{(}\PY{n}{a} \PY{o}{\PYZgt{}} \PY{l+m+mi}{0}\PY{p}{,} \PY{n}{a}\PY{p}{,} \PY{l+m+mf}{1e\PYZhy{}300}\PY{p}{)}\PY{p}{)} \PY{o}{\PYZhy{}} \PY{n}{mu} \PY{o}{\PYZhy{}} \PY{n}{k} \PY{o}{*} \PY{p}{(}\PY{n}{sigma} \PY{o}{*}\PY{o}{*} \PY{l+m+mi}{2}\PY{p}{)}\PY{p}{)} \PY{o}{/} \PY{n}{sigma}
    \PY{n}{I} \PY{o}{=} \PY{n}{pref\PYZus{}I} \PY{o}{*} \PY{n}{\PYZus{}Phi}\PY{p}{(}\PY{n}{z}\PY{p}{)}
    \PY{n}{I}\PY{p}{[}\PY{n}{a} \PY{o}{\PYZlt{}}\PY{o}{=} \PY{l+m+mi}{0}\PY{p}{]} \PY{o}{=} \PY{l+m+mf}{0.0}
    \PY{k}{return} \PY{n}{\PYZus{}PREF} \PY{o}{*} \PY{n}{np}\PY{o}{.}\PY{n}{where}\PY{p}{(}\PY{n}{f} \PY{o}{\PYZgt{}} \PY{l+m+mi}{0}\PY{p}{,} \PY{n}{f} \PY{o}{*}\PY{o}{*} \PY{o}{\PYZhy{}}\PY{l+m+mf}{2.5}\PY{p}{,} \PY{l+m+mf}{0.0}\PY{p}{)} \PY{o}{*} \PY{n}{I}

\PY{k}{def}\PY{+w}{ }\PY{n+nf}{pdf\PYZus{}flux\PYZus{}pareto\PYZus{}alpha2}\PY{p}{(}\PY{n}{f}\PY{p}{,} \PY{n}{xm}\PY{o}{=}\PY{l+m+mf}{1.0}\PY{p}{)}\PY{p}{:}
    \PY{n}{f} \PY{o}{=} \PY{n}{np}\PY{o}{.}\PY{n}{asarray}\PY{p}{(}\PY{n}{f}\PY{p}{,} \PY{n}{dtype} \PY{o}{=} \PY{n+nb}{float}\PY{p}{)}
    \PY{n}{p} \PY{o}{=} \PY{n}{np}\PY{o}{.}\PY{n}{zeros\PYZus{}like}\PY{p}{(}\PY{n}{f}\PY{p}{)}
    \PY{n}{fmin} \PY{o}{=} \PY{n}{xm} \PY{o}{/} \PY{p}{(}\PY{l+m+mf}{4.0} \PY{o}{*} \PY{n}{pi}\PY{p}{)}
    \PY{n}{mask} \PY{o}{=} \PY{n}{f} \PY{o}{\PYZgt{}}\PY{o}{=} \PY{n}{fmin}
    \PY{n}{factor} \PY{o}{=} \PY{l+m+mf}{4.0} \PY{o}{*} \PY{n}{xm} \PY{o}{*}\PY{o}{*} \PY{l+m+mi}{2}
    \PY{c+c1}{\PYZsh{} (1 \PYZhy{} (4*pi*f)\PYZca{}\PYZhy{}0.5) comes from integrating the Pareto luminosity; keep safe numeric ops}
    \PY{n}{p}\PY{p}{[}\PY{n}{mask}\PY{p}{]} \PY{o}{=} \PY{n}{\PYZus{}PREF} \PY{o}{*} \PY{p}{(}\PY{n}{f}\PY{p}{[}\PY{n}{mask}\PY{p}{]} \PY{o}{*}\PY{o}{*} \PY{o}{\PYZhy{}}\PY{l+m+mf}{2.5}\PY{p}{)} \PY{o}{*} \PY{n}{factor} \PY{o}{*} \PY{p}{(}\PY{l+m+mf}{1.0} \PY{o}{\PYZhy{}} \PY{p}{(}\PY{l+m+mf}{4.0} \PY{o}{*} \PY{n}{pi} \PY{o}{*} \PY{n}{f}\PY{p}{[}\PY{n}{mask}\PY{p}{]}\PY{p}{)} \PY{o}{*}\PY{o}{*} \PY{o}{\PYZhy{}}\PY{l+m+mf}{0.5}\PY{p}{)}
    \PY{k}{return} \PY{n}{p}

\PY{c+c1}{\PYZsh{} (Optional) overlay theoretical curves for each panel for visual check}
\PY{n}{f\PYZus{}vals} \PY{o}{=} \PY{n}{np}\PY{o}{.}\PY{n}{logspace}\PY{p}{(}\PY{n}{np}\PY{o}{.}\PY{n}{log10}\PY{p}{(}\PY{n}{np}\PY{o}{.}\PY{n}{nanmax}\PY{p}{(}\PY{p}{[}\PY{n}{np}\PY{o}{.}\PY{n}{min}\PY{p}{(}\PY{n}{fluxes}\PY{p}{[}\PY{n}{i}\PY{p}{]}\PY{p}{[}\PY{n}{fluxes}\PY{p}{[}\PY{n}{i}\PY{p}{]} \PY{o}{\PYZgt{}} \PY{l+m+mi}{0}\PY{p}{]}\PY{p}{)} \PY{k}{for} \PY{n}{i} \PY{o+ow}{in} \PY{n+nb}{range}\PY{p}{(}\PY{l+m+mi}{4}\PY{p}{)}\PY{p}{]}\PY{p}{)}\PY{p}{)} \PY{o}{\PYZhy{}} \PY{l+m+mi}{1}\PY{p}{,}
                     \PY{n}{np}\PY{o}{.}\PY{n}{log10}\PY{p}{(}\PY{n}{np}\PY{o}{.}\PY{n}{nanmax}\PY{p}{(}\PY{p}{[}\PY{n}{np}\PY{o}{.}\PY{n}{max}\PY{p}{(}\PY{n}{fluxes}\PY{p}{[}\PY{n}{i}\PY{p}{]}\PY{p}{)} \PY{k}{for} \PY{n}{i} \PY{o+ow}{in} \PY{n+nb}{range}\PY{p}{(}\PY{l+m+mi}{4}\PY{p}{)}\PY{p}{]}\PY{p}{)}\PY{p}{)} \PY{o}{+} \PY{l+m+mi}{1}\PY{p}{,} \PY{l+m+mi}{200}\PY{p}{)}

\PY{c+c1}{\PYZsh{} Avoid potential issues if any flux array has zeros or negative values}
\PY{n}{f\PYZus{}vals} \PY{o}{=} \PY{n}{np}\PY{o}{.}\PY{n}{clip}\PY{p}{(}\PY{n}{f\PYZus{}vals}\PY{p}{,} \PY{l+m+mf}{1e\PYZhy{}300}\PY{p}{,} \PY{k+kc}{None}\PY{p}{)}

\PY{c+c1}{\PYZsh{} Plot theoretical overlays (choose appropriate parameters for each luminosity function)}
\PY{n}{ax}\PY{p}{[}\PY{l+m+mi}{0}\PY{p}{]}\PY{o}{.}\PY{n}{plot}\PY{p}{(}
    \PY{n}{f\PYZus{}vals}\PY{p}{,}
    \PY{n}{pdf\PYZus{}flux\PYZus{}const}\PY{p}{(}\PY{n}{f\PYZus{}vals}\PY{p}{,} \PY{l+m+mf}{10.0}\PY{p}{)}\PY{p}{,}
    \PY{l+s+s1}{\PYZsq{}}\PY{l+s+s1}{r\PYZhy{}}\PY{l+s+s1}{\PYZsq{}}\PY{p}{,}
    \PY{n}{label} \PY{o}{=} \PY{l+s+s1}{\PYZsq{}}\PY{l+s+s1}{Theory}\PY{l+s+s1}{\PYZsq{}}
\PY{p}{)}
\PY{n}{ax}\PY{p}{[}\PY{l+m+mi}{1}\PY{p}{]}\PY{o}{.}\PY{n}{plot}\PY{p}{(}
    \PY{n}{f\PYZus{}vals}\PY{p}{,}
    \PY{n}{pdf\PYZus{}flux\PYZus{}uniform}\PY{p}{(}\PY{n}{f\PYZus{}vals}\PY{p}{,} \PY{l+m+mf}{0.0}\PY{p}{,} \PY{l+m+mf}{20.0}\PY{p}{)}\PY{p}{,}
    \PY{l+s+s1}{\PYZsq{}}\PY{l+s+s1}{r\PYZhy{}}\PY{l+s+s1}{\PYZsq{}}\PY{p}{,}
    \PY{n}{label} \PY{o}{=} \PY{l+s+s1}{\PYZsq{}}\PY{l+s+s1}{Theory}\PY{l+s+s1}{\PYZsq{}}
\PY{p}{)}
\PY{n}{ax}\PY{p}{[}\PY{l+m+mi}{2}\PY{p}{]}\PY{o}{.}\PY{n}{plot}\PY{p}{(}
    \PY{n}{f\PYZus{}vals}\PY{p}{,}
    \PY{n}{pdf\PYZus{}flux\PYZus{}lognormal}\PY{p}{(}\PY{n}{f\PYZus{}vals}\PY{p}{,} \PY{n}{mu} \PY{o}{=} \PY{l+m+mf}{2.2}\PY{p}{,} \PY{n}{sigma} \PY{o}{=} \PY{l+m+mf}{0.3}\PY{p}{)}\PY{p}{,}
    \PY{l+s+s1}{\PYZsq{}}\PY{l+s+s1}{r\PYZhy{}}\PY{l+s+s1}{\PYZsq{}}\PY{p}{,}
    \PY{n}{label} \PY{o}{=} \PY{l+s+s1}{\PYZsq{}}\PY{l+s+s1}{Theory}\PY{l+s+s1}{\PYZsq{}}
\PY{p}{)}
\PY{n}{ax}\PY{p}{[}\PY{l+m+mi}{3}\PY{p}{]}\PY{o}{.}\PY{n}{plot}\PY{p}{(}
    \PY{n}{f\PYZus{}vals}\PY{p}{,}
    \PY{n}{pdf\PYZus{}flux\PYZus{}pareto\PYZus{}alpha2}\PY{p}{(}\PY{n}{f\PYZus{}vals}\PY{p}{,} \PY{n}{xm} \PY{o}{=} \PY{l+m+mf}{1.0}\PY{p}{)}\PY{p}{,}
    \PY{l+s+s1}{\PYZsq{}}\PY{l+s+s1}{r\PYZhy{}}\PY{l+s+s1}{\PYZsq{}}\PY{p}{,}
    \PY{n}{label} \PY{o}{=} \PY{l+s+s1}{\PYZsq{}}\PY{l+s+s1}{Theory}\PY{l+s+s1}{\PYZsq{}}
\PY{p}{)}

\PY{k}{for} \PY{n}{a} \PY{o+ow}{in} \PY{n}{ax}\PY{p}{:}
    \PY{n}{a}\PY{o}{.}\PY{n}{set\PYZus{}xlim}\PY{p}{(}\PY{o}{\PYZhy{}}\PY{l+m+mi}{1}\PY{p}{,} \PY{l+m+mi}{200}\PY{p}{)}
    \PY{n}{a}\PY{o}{.}\PY{n}{set\PYZus{}yscale}\PY{p}{(}\PY{l+s+s1}{\PYZsq{}}\PY{l+s+s1}{log}\PY{l+s+s1}{\PYZsq{}}\PY{p}{)}
    \PY{n}{a}\PY{o}{.}\PY{n}{set\PYZus{}xlabel}\PY{p}{(}\PY{l+s+s1}{\PYZsq{}}\PY{l+s+s1}{Flux}\PY{l+s+s1}{\PYZsq{}}\PY{p}{)}
    \PY{n}{a}\PY{o}{.}\PY{n}{set\PYZus{}ylabel}\PY{p}{(}\PY{l+s+s1}{\PYZsq{}}\PY{l+s+s1}{Probability Density}\PY{l+s+s1}{\PYZsq{}}\PY{p}{)}
    \PY{n}{a}\PY{o}{.}\PY{n}{legend}\PY{p}{(}\PY{p}{)}

\PY{n}{plt}\PY{o}{.}\PY{n}{tight\PYZus{}layout}\PY{p}{(}\PY{p}{)}
\PY{n}{plt}\PY{o}{.}\PY{n}{show}\PY{p}{(}\PY{p}{)}
\end{Verbatim}
\end{tcolorbox}

    \begin{center}
    \adjustimage{max size={0.9\linewidth}{0.9\paperheight}}{Exercise_week_03_files/Exercise_week_03_38_0.png}
    \end{center}
    { \hspace*{\fill} \\}
    
    \subsubsection{Note: steps for submitting the
exercise}\label{note-steps-for-submitting-the-exercise}

\begin{enumerate}
\def\labelenumi{\arabic{enumi}.}
\tightlist
\item
  In the menu bar, select \texttt{File\ \textgreater{}\ Download} to
  download your notebook as a \texttt{.ipynb} file.
\item
  Select
  \texttt{File\ \textgreater{}\ Save\ and\ Export\ Notebook\ As\ \textgreater{}\ PDF}
  to export your notebook as a PDF file.
\item
  Combine the \texttt{.ipynb} and \texttt{.pdf} files into a single
  \texttt{.zip} or \texttt{.tar.gz} archive.
\item
  Upload your archive to the web learning platform (网络学堂).
\end{enumerate}


    % Add a bibliography block to the postdoc
    
    
    
\end{document}
